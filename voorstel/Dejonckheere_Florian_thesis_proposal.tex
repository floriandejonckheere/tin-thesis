%==============================================================================
% Sjabloon onderzoeksvoorstel bachelorproef
%==============================================================================
% Gebaseerd op LaTeX-sjabloon ‘Stylish Article’ (zie voorstel.cls)
% Auteur: Jens Buysse, Bert Van Vreckem

% TODO: Compileren document:
% 1) Vervang ‘naam_voornaam’ in de bestandsnaam door je eigen naam, bv.
%    buysse_jens_voorstel.tex
% 2) latexmk -pdf naam_voornaam_voorstel.tex
% 3) biber naam_voornaam_voorstel
% 4) latexmk -pdf naam_voornaam_voorstel.tex (1 keer)

\documentclass[fleqn,10pt]{voorstel}

%------------------------------------------------------------------------------
% Metadata over het artikel
%------------------------------------------------------------------------------

\JournalInfo{HoGent Bedrijf en Organisatie} % Journal information
\Archive{Bachelor's thesis 2017 - 2018} % Additional notes (e.g. copyright, DOI, review/research article)

%---------- Titel & auteur ----------------------------------------------------

% TODO: geef werktitel van je eigen voorstel op
\PaperTitle{Analysis of efficient graph data storage mechanisms for a structured social notification feed}
\PaperType{Research thesis proposal} % Type document

% TODO: vul je eigen naam in als auteur, geef ook je emailadres mee!
\Authors{Florian Dejonckheere} % Authors
\affiliation{\textbf{Contact:} \href{mailto:florian@floriandejonckheere.be}{florian@floriandejonckheere.be}}

%---------- Abstract ----------------------------------------------------------

\Abstract{

    % Context (why is this work important)
    % Demand (why is it necessary to research this)
    % Task (what is going to happen)
    % Object (what is written here)
    % Result (what is expected)
    % Conclusion
    % Perspective

    Ever since the dawn of dynamic web applications and social networks, there has been an increasing demand for performant storage systems to capture the vast deluge of data that is generated by the social activity of users. This data is characterised by its size and inherently structured nature: most types of social networks can be modelled using certain predetermined graph topologies. In this paper, the current state of affairs pertaining to the storage of data in small to medium-size web applications will be summarized. An analysis of the structured data within the context of the Open Webslides project will be made, and an attempt will be made to match this analysis to a stable, scalable and maintained storage system. A proposed implementation plan for a normalized, efficient data schema will also be presented.
}

%---------- Onderzoeksdomein en sleutelwoorden --------------------------------
% TODO: Sleutelwoorden:
%
% Het eerste sleutelwoord beschrijft het onderzoeksdomein. Je kan kiezen uit
% deze lijst:
%
% - Mobiele applicatieontwikkeling
% - Webapplicatieontwikkeling
% - Applicatieontwikkeling (andere)
% - Systeem- en netwerkbeheer
% - Mainframe
% - E-business
% - Databanken en big data
% - Machine learning en kunstmatige intelligentie
% - Andere (specifieer)
%
% De andere sleutelwoorden zijn vrij te kiezen

\Keywords{Webapplications. Graph --- Social feed --- Data storage --- Cloud --- NoSQL} % Keywords
\newcommand{\keywordname}{Keywords} % Defines the keywords heading name

%---------- Titel, inhoud -----------------------------------------------------
\begin{document}

\flushbottom % Makes all text pages the same height
\maketitle % Print the title and abstract box
\tableofcontents % Print the contents section
\thispagestyle{empty} % Removes page numbering from the first page

%------------------------------------------------------------------------------
% Hoofdtekst
%------------------------------------------------------------------------------

%---------- Inleiding ---------------------------------------------------------

\section{Introduction} % The \section*{} command stops section numbering
\label{sec:introduction}

% problem and context
% motivation, relevance for the research
% goal and research questions

The Open Webslides project \autocite{OpenWebslides} provides a user-friendly platform to collaborate on webslides - slides made with modern web technologies such as HTML, CSS and JavaScript. One of the core features this application provides is \emph{co-creation}. The co-creation aspect manifests itself in several forms within the application; annotations on slides and a change suggesting system resembling GitHub's pull request feature are the main mechanisms. Because of the inherent social nature of co-creation, a basic notifications feed was also implemented. This feed is tailored to the user, and reflects the most recent changes, additions and comments relevant to the slide decks the user is interested in.

However, the functionality implemented in the system contains only the bare necessities at the moment. The module will be expanded in the future, and doing so requires a structural and conceptual rethinking of how the notifications are generated, stored and queried. This paper has two concrete goals: First, it aims to analyze and summarize the existing frameworks and software packages commonly used in the industry to store structured non-relational graph or document data.

Second, the structure and data provided by the Open Webslides' social notification feed will be interpreted in the context of the aforementioned analysis. Finally, a recommendation will be made for a concrete implementation of the described data storage schema.

%---------- Stand van zaken ---------------------------------------------------

\section{State of the art}
\label{sec:state-of-the-art}

In current literature, many existing studies have already described the relation between traditional relational database systems and NoSQL stores. However, since this paper covers a specific use case no further general comparative studies will be referenced.

The 2015 doctoral thesis \autocite{Zhao2015} describes the development of a messaging system for astrophysical transient event notifications. Part of this analysis is a comparison between document-based NoSQL storage solutions fit for this particular use case. We expect this paper to provide a solid base of reasoning in order to find a scalable and efficient solution for resolving similar computational challenges.

The main difference between the previously mentioned studies and this paper is the specific use case. This paper tries to present a solution tailored to the specific requirements of the Open Webslides project. This entails a different handling of certain key data-structures within the provided platform, particularly concerning querying the stored data.

% \autocite{KEY} => (Author, year)
% \textcite{KEY} => Author (year)

%---------- Methodologie ------------------------------------------------------
\section{Methodology}
\label{sec:methodology}

Hier beschrijf je hoe je van plan bent het onderzoek te voeren. Welke onderzoekstechniek ga je toepassen om elk van je onderzoeksvragen te beantwoorden? Gebruik je hiervoor experimenten, vragenlijsten, simulaties? Je beschrijft ook al welke tools je denkt hiervoor te gebruiken of te ontwikkelen.

% Conceptual: comparison of NoSQL dbs, (dis-)advantages
% Conceptual: analysis of structural data + write once read many + highly linked
% Empirical testing: sample data set with users, notifications, etc. tested against multiple NoSQL database queries
% Comparison of (non-)relational databases
%   - No CAP needed
% Scalability: horizontal, vertical + expected load (# users)

%---------- Verwachte resultaten ----------------------------------------------
\section{Expected results}
\label{sec:expected_results}

Hier beschrijf je welke resultaten je verwacht. Als je metingen en simulaties uitvoert, kan je hier al mock-ups maken van de grafieken samen met de verwachte conclusies. Benoem zeker al je assen en de stukken van de grafiek die je gaat gebruiken. Dit zorgt ervoor dat je concreet weet hoe je je data gaat moeten structureren.

% The  NoSQL ecosystem, unlike relational databases, is headed towards specialization, so different solutions are headed in different directions
%   => we expect to find one or two products that fit our use case better

%---------- Verwachte conclusies ----------------------------------------------
\section{Expected conclusions}
\label{sec:expected_conclusions}

Hier beschrijf je wat je verwacht uit je onderzoek, met de motivatie waarom. Het is \textbf{niet} erg indien uit je onderzoek andere resultaten en conclusies vloeien dan dat je hier beschrijft: het is dan juist interessant om te onderzoeken waarom jouw hypothesen niet overeenkomen met de resultaten.

%------------------------------------------------------------------------------
% Referentielijst
%------------------------------------------------------------------------------
% TODO: de gerefereerde werken moeten in BibTeX-bestand ``biblio.bib''
% voorkomen. Gebruik JabRef om je bibliografie bij te houden en vergeet niet
% om compatibiliteit met Biber/BibLaTeX aan te zetten (File > Switch to
% BibLaTeX mode)

\phantomsection
\printbibliography[heading=bibintoc]

\end{document}
