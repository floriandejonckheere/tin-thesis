%%=============================================================================
%% LaTeX sjabloon voor bachelorproef, HoGent Bedrijf en Organisatie
%% Opleiding Toegepaste Informatica
%%=============================================================================

\documentclass[fleqn,a4paper,12pt]{book}

\input{structure}

%%---------- Documenteigenschappen --------------------------------------------
%% TODO: Vul dit aan met je eigen info:

% Je eigen naam
\newcommand{\student}{Florian Dejonckheere}

% De naam van je promotor (lector van de opleiding)
\newcommand{\promotor}{Chantal Teerlinck}

% De naam van je co-promotor. Als je promotor ook je opdrachtgever is en je
% dus ook inhoudelijk begeleidt (en enkel dan!), mag je dit leeg laten.
\newcommand{\copromotor}{}

% Indien je bachelorproef in opdracht van/in samenwerking met een bedrijf of
% externe organisatie geschreven is, geef je hier de naam. Zoniet laat je dit
% zoals het is.
\newcommand{\instelling}{Open Webslides}

% De titel van het rapport/bachelorproef
\newcommand{\titel}{Comparative Study of NoSQL Data Storage Solutions for an Interlinked Recent Activity Feed}

% Datum van indienen (gebruik telkens de deadline, ook al geef je eerder af)
\newcommand{\datum}{28 May 2018}

% Academiejaar
\newcommand{\academiejaar}{2017-2018}

% Examenperiode
%  - 1e semester = 1e examenperiode => 1
%  - 2e semester = 2e examenperiode => 2
%  - tweede zit  = 3e examenperiode => 3
\newcommand{\examenperiode}{2}

%%=============================================================================
%% Inhoud document
%%=============================================================================

\begin{document}

%---------- Taalselectie ------------------------------------------------------
% Als je je bachelorproef in het Engels schrijft, haal dan onderstaande regel
% uit commentaar. Let op: de tekst op de voorkaft blijft in het Nederlands, en
% dat is ook de bedoeling!

\selectlanguage{english}

%---------- Titelblad ---------------------------------------------------------
\inserttitlepage

%---------- Samenvatting, voorwoord -------------------------------------------
\usechapterimagefalse
%%=============================================================================
%% Preface
%%=============================================================================

\chapter*{Preface}
\label{ch:preface}

%% TODO:
%% Het voorwoord is het enige deel van de bachelorproef waar je vanuit je
%% eigen standpunt (``ik-vorm'') mag schrijven. Je kan hier bv. motiveren
%% waarom jij het onderwerp wil bespreken.
%% Vergeet ook niet te bedanken wie je geholpen/gesteund/... heeft

The idea for this research originally comes from the Open Webslides project and its many little side activities in development.
One of the things that has always fascinated me was how the platform would handle a massive influx of users, and specifically how it would relate to the non-critical data storage in the Recent Activity feed.
I wanted to find out how a NoSQL data store would be integrated into the flow of information, and what kind of data store would be the most efficient, scalable solution for this problem.
My interest was also piqued by the usage of a graph database (Neo4j) in a personal project, and how the data of the Open Webslides project would fit into the graph theoretical model.
Digging into this subject while still maintaining my vision on the Ruby on Rails implementation in the platform allowed me to let the question bloom into this research thesis.

This thesis was in part achieved by the support of my promotor, who has given me many tips and tricks, and provided a framework for conducting a proper research.
My co-promotor also had an important influence on certain decisions taken in the research and development phase, being a person who is immersed in the relational and non-relational academic database world.
Finally, my friends and family also deserve recognition for helping me accomplish this paper, which is the culmination of three years higher education in a fast-moving and innovative field.

%%=============================================================================
%% Summary
%%=============================================================================

% TODO: De "abstract" of samenvatting is een kernachtige (~ 1 blz. voor een
% thesis) synthese van het document.
%
% Deze aspecten moeten zeker aan bod komen:
% - Context: waarom is dit werk belangrijk?
% - Nood: waarom moest dit onderzocht worden?
% - Taak: wat heb je precies gedaan?
% - Object: wat staat in dit document geschreven?
% - Resultaat: wat was het resultaat?
% - Conclusie: wat is/zijn de belangrijkste conclusie(s)?
% - Perspectief: blijven er nog vragen open die in de toekomst nog kunnen
%    onderzocht worden? Wat is een mogelijk vervolg voor jouw onderzoek?
%
% LET OP! Een samenvatting is GEEN voorwoord!

%%---------- Nederlandse samenvatting -----------------------------------------
%
% TODO: Als je je bachelorproef in het Engels schrijft, moet je eerst een
% Nederlandse samenvatting invoegen. Haal daarvoor onderstaande code uit
% commentaar.
% Wie zijn bachelorproef in het Nederlands schrijft, kan dit negeren, de inhoud
% wordt niet in het document ingevoegd.

\IfLanguageName{english}{%
\selectlanguage{dutch}
\chapter*{Samenvatting}


\selectlanguage{english}
}{}

%%---------- Samenvatting -----------------------------------------------------
% De samenvatting in de hoofdtaal van het document

\chapter*{\IfLanguageName{dutch}{Samenvatting}{Abstract}}

The advent of large scale, dynamic web applications has led to a massive increase in the need for performant database systems to store the deluge of information generated by user activity.
The industry's answer to this problem is the NoSQL movement, which prioritizes availability and scalability over traditional concerns such as data consistency and reliability.
An increasingly interesting question for researchers and developers in the field is how to store this data in order to maximize the query performance, considering the use case and the inherent structure of the data involved.

This thesis dives into the world of NoSQL and non-relational data modeling, comparing and examining the different NoSQL data store types and vendors.
The Recent Activity feed of the Open Webslides platform is used as a case study.
For the \TODO{five} categories of NoSQL data stores, a comparative study is presented in which attributes relevant to the use case are compared.
Subsequently, two logical data models for document and graph data stores are presented, along with three corresponding implementations for the MongoDB, CouchDB and Neo4j data stores.

The research concluded that document stores are the most efficient data stores among the solutions considered.
The NoSQL data stores are to be used complementary with a relational database, leveraging polyglot persistence to achieve a performant and scalable web application.

The NoSQL implementations developed in this thesis will provide capable and powerful solutions for the problem presented in the case study.
However, it was found that there are many opportunities to extend this research beyond the initial use case to allow additional contributions to the field.


%---------- Inhoudstafel ------------------------------------------------------
\pagestyle{empty} % No headers
\tableofcontents % Print the table of contents itself
\cleardoublepage % Forces the first chapter to start on an odd page so it's on the right
\pagestyle{fancy} % Print headers again

%---------- List of abbreviations and glossary --------------------------------

\printglossary
\printglossary[type=\acronymtype,title={List of abbreviations}]


%---------- Lijst figuren, afkortingen, ... -----------------------------------

% Indien gewenst kan je hier een lijst van figuren/tabellen opgeven. Geef in
% dat geval je figuren/tabellen altijd een korte beschrijving:
%
%  \caption[korte beschrijving]{uitgebreide beschrijving}

\listoffigures
\listoftables

% Als je een lijst van afkortingen of termen wil toevoegen, dan hoort die
% hier thuis. Gebruik bijvoorbeeld de ``glossaries'' package.
% https://www.sharelatex.com/learn/Glossaries

%%---------- Kern -------------------------------------------------------------

%%=============================================================================
%% Inleiding
%%=============================================================================

\chapter*{Introduction}
\label{ch:introduction}

De inleiding moet de lezer net genoeg informatie verschaffen om het onderwerp te begrijpen en in te zien waarom de onderzoeksvraag de moeite waard is om te onderzoeken. In de inleiding ga je literatuurverwijzingen beperken, zodat de tekst vlot leesbaar blijft. Je kan de inleiding verder onderverdelen in secties als dit de tekst verduidelijkt. Zaken die aan bod kunnen komen in de inleiding~\autocite{Pollefliet2011}:

\begin{itemize}
  \item context, achtergrond
  \item afbakenen van het onderwerp
  \item verantwoording van het onderwerp, methodologie
  \item probleemstelling
  \item onderzoeksdoelstelling
  \item onderzoeksvraag
  \item \ldots
\end{itemize}

\section{Context}
\label{sec:context}

In this age of computers and smartphones, older and deprecated methods of teach are quickly being replaced by the digital equivalent. Course content has shifted from being printed in full-text on paper, to static slides on an overhead projector and to the digital screen present in every modern classroom. However, there is a lot more potential to gain from the modernization of course content. Education is still too often a one-way street, where students are obligated to process the course content without being able to provide much challenge or activity. This also does not allow for any dialogue to take place between students and teachers concerning feedback and improvement of the material itself.

Technological constraints in current iterations of educational software do not allow this co-creation discourse easily. Material being locked to specific versions of proprietary software is just one of the many problems teachers might encounter when trying to apply this concept in real life.

By utilizing interactive tools and applications, the teacher can engage the students more directly.

By building on modern, open standards, the Open Webslides project \textcite{OpenWebslides2017} aims to provide a platform that solves these problems. It creates a user-friendly environment where teachers can create courses based on open source technologies and standards, and it allows them to apply the co-creation narrative easily. This also enables users to share their material not just with their immediate environment but with a much broader educational audience.

\section{Problem statement}
\label{sec:problem-statement}

Uit je probleemstelling moet duidelijk zijn dat je onderzoek een meerwaarde heeft voor een concrete doelgroep. De doelgroep moet goed gedefinieerd en afgelijnd zijn. Doelgroepen als ``bedrijven,'' ``KMO's,'' systeembeheerders, enz.~zijn nog te vaag. Als je een lijstje kan maken van de personen/organisaties die een meerwaarde zullen vinden in deze bachelorproef (dit is eigenlijk je steekproefkader), dan is dat een indicatie dat de doelgroep goed gedefinieerd is. Dit kan een enkel bedrijf zijn of zelfs één persoon (je co-promotor/opdrachtgever).

\section{Research question}
\label{sec:research-question}

Wees zo concreet mogelijk bij het formuleren van je onderzoeksvraag. Een onderzoeksvraag is trouwens iets waar nog niemand op dit moment een antwoord heeft (voor zover je kan nagaan). Het opzoeken van bestaande informatie (bv. ``welke tools bestaan er voor deze toepassing?'') is dus geen onderzoeksvraag. Je kan de onderzoeksvraag verder specifiëren in deelvragen. Bv.~als je onderzoek gaat over performantiemetingen, dan

\section{Research goal}
\label{sec:research-goal}

Wat is het beoogde resultaat van je bachelorproef? Wat zijn de criteria voor succes? Beschrijf die zo concreet mogelijk.

\section{Thesis objective}
\label{sec:thesis-objective}

% Het is gebruikelijk aan het einde van de inleiding een overzicht te
% geven van de opbouw van de rest van de tekst. Deze sectie bevat al een aanzet
% die je kan aanvullen/aanpassen in functie van je eigen tekst.

De rest van deze bachelorproef is als volgt opgebouwd:

In Hoofdstuk~\ref{ch:state-of-the-art} wordt een overzicht gegeven van de stand van zaken binnen het onderzoeksdomein, op basis van een literatuurstudie.

In Hoofdstuk~\ref{ch:methodology} wordt de methodologie toegelicht en worden de gebruikte onderzoekstechnieken besproken om een antwoord te kunnen formuleren op de onderzoeksvragen.

% TODO: Vul hier aan voor je eigen hoofstukken, één of twee zinnen per hoofdstuk

In Hoofdstuk~\ref{ch:conclusion}, tenslotte, wordt de conclusie gegeven en een antwoord geformuleerd op de onderzoeksvragen. Daarbij wordt ook een aanzet gegeven voor toekomstig onderzoek binnen dit domein.


\chapter{State of the Art}
\label{ch:state-of-the-art}

%%
%% Introduction
%%
Ever since the rise of the NoSQL databases in 2009 \autocite{Sadalage2012} it has been a subject for vigorous academic and professional research. The contrast with relational databases, optimal use cases, performance and scalability are only some of the aspects that have been analyzed with great regularity. This chapter will summarize previous publications relevant to this thesis.

%%
%% Literature study
%%

% NoSQL Distilled: A Brief Guide to the Emerging World of Polyglot Persistence
The book by \textcite{Sadalage2012} provides an excellent entry into the world of NoSQL. It explains the motivation behind the use of NoSQL techniques, and how this differs from relational data storage. Furthermore it introduced the segregation of NoSQL data stores into four main categories: key-value, document, column and graph databases.
Second, \citeauthor{Sadalage2012} touch the concept of polyglot persistence. This discribes the application's opportunity to use multiple types of data stores to store heterogeneously structured data. This technique is relevant in particular to this thesis, as the describe data schema only relates to one of the database management systems integrated in the Open Webslides platform.
The second part of the book provides a more practical approach to using polyglot persistence in an enterprise application. The authors have written down many pointers and guides in order to pick the right database for the use case.

% Type of NOSQL Databases and its Comparison with Relational Databases
% NoSQL Database: New Era of Databases for Big data Analytics - Classification, Characteristics and Comparison
% Scalable SQL and NoSQL Data Stores
There have already been numerous studies to differentiate the different types of NoSQL databases and comparative studies between NoSQL database systems. \textcite{Nayak2013} presents a fifth NoSQL category: object-oriented databases together with a comparative study of NoSQL data stores.
Other studies such as \textcite{Moniruzzaman2013} provide a feature-based comparison for various NoSQL vendors and database systems.
Finally, the similarity and resemblance of relational and NoSQL data stores is also a well-researched topic in current literature. Studies and surveys such as \textcite{Mohamed2014} and \textcite{Cattell2010} tackle this subject in great detail.

% Data Management in Cloud Environments: NoSQL and NewSQL Data Stores
\textcite{Grolinger2013} present a use-case based approach to comparing different NoSQL and NewSQL data stores. The survey incorporates a feature-based comparison over different aspects such as querying, scalability and security, and analyzes these concepts in the context of a select number of NoSQL data stores.

% NoSQL evaluation: A Use Case Oriented Survey
\textcite{Hecht2011} provides a feature-based comparison of different NoSQL database types and vendors. The researchers compare the data model, querying access, concurrency, partioning and replication. They use a duality-based approach, where a minus indicates that the feature is not supported by the database system, and a plus if the feature is supported. The paper also presents the problem of a lack of unified querying interface for NoSQL databases. Furthermore, the importance of choosing the right NoSQL database type for the use case is emphasized, however Hecht and Jablonski do not present a specific case study.

% Modeling and Querying Data in NoSQL Databases
The proceedings of the 2013 IEEE International Conference on Big data by \textcite{Kaur2013} describe the theoretical modeling and querying of SQL and NoSQL data stores. The paper then proceeds with a case study of a social networking site similar to Slashdot \autocite{Malda1997}. Starting from an entity-relationship diagram (ERD), the researchers then proceed by modeling the entities in both a document and a graph database. Finally, a set of seven queries related to the use case is then drawn up and compared for the PostgreSQL, MongoDB and Neo4j data stores.

% Event Based Transient Notification Architecture and NoSQL Solution for Astronomical Data Management
\textcite{Zhao2015} explores the use of NoSQL data stores to store huge amounts of observational data generated by astronomical research. It briefly discusses using filesystems and relational data stores, before comparing NoSQL alternatives. A concrete data model to store the astronomical data in a MongoDB data store is then presented, together with eight scenarios and queries that may be used in a production system.
Furthermore, performance measurements of MongoDB are also analyzed. Data insertion, querying and deletion using the aforementioned data scheme and real observational data are used in this section.

% Performance Investigation of Selected SQL and NoSQL Databases
The proceedings of the AGILE 2015 conference by \textcite{Schmid2015} present an overview of selected SQL and NoSQL databases, focusing on the geo-functionalities of the systems. It uses performance tests between two document-based NoSQL data stores (MongoDB and CouchBase). The researchers conclude that geospatial calculations in NoSQL database systems are still only supported for basic queries. Relational databases still perform superior to NoSQL databases in small to larger data sets for queries with geo-functions. However the NoSQL response time only increases slightly relative to data set size.

% On Scalability of Two NoSQL Data Stores for Processing Interactive Social Networking Actions
The technical report by \textcite{Barahmand2015} quantifies the scalability of MongoDB and HBase for processing simple operations using the social networking benchmark BG \autocite{Barahmand2013}. It considers both horizontal and vertical scalability of the data stores using the Social Action Rating (SoAR) introduced by the benchmarking tool.
In order to perform these benchmarks, two logical data models for the database design are presented. The report concludes that while both data stores scale superlinearly, their speedup is limited by the resources of a few nodes out of many becoming fully utilized.

% Which NoSQL Database? A Performance Overview
Another performance-based study written by \textcite{Abramova2014} compares five popular NoSQL databases (Cassandra, HBase, MongoDB, OrientDB and Redis) using the Yahoo! Cloud Serving Benchmark \autocite{Yahoo2010}. The study compares read and write query performance. It concludes that over the five compared data stores, MongoDB, Redis and OrientDB are more read-optimized, and Cassandra and HBase are more update-optimized.

% A Survey of Data Management System for Cloud Computing: Models and Searching Methods
A more query-oriented study was performed by \textcite{Zhou2013}. The paper considers both the academic and the industry definition and description of data models and system architectures. The researchers identify two kinds of searching approaches: the MapReduce-oriented and the SQL-like querying.

% Data Modeling in the NoSQL World
The work of \textcite{Atzeni2016} dives deeper into the world of non-relational data modeling. The paper investigates how traditional data modeling can be used in the context of schemaless and heterogeneous data stores. \citeauthor{Atzeni2016} propose NoAM (NoSQL Abstract Modeling), an abstract data model to describe NoSQL databases based on the common surfaces of the various data store types. This technique can be used to describe system-independent application data and later to implement this in the specific data stores, taking advantage of the various target system idiosyncrasies.

%%
%% Difference with related work
%%

The main differences between this study and the previous studies are:

\begin{enumerate}
  \item Many studies have been conducted to understand the motivation between the NoSQL principles and the shift from relational data stores. The division of NoSQL data store types into four categories and elaboration upon this is usually also a topic in these studies. This research paper builds upon that knowledge, providing only a brief introduction in the world of NoSQL and NewSQL concepts.
  \item Some of the previously mentioned research papers also discuss a case study applied to a specific use case. This is mostly related to business critical systems that store and process large volumes of data. The use case described in this thesis is very specific in that it's a complementary subsystem that does not affect critical data. Therefor, certain comparative attributes such as security and availability are not considered in this research.
\end{enumerate}

%%
%% Added value of thesis
%%

This thesis aims to provide a case study of data storage the \textcite{OpenWebslides2017} platform. Several use case based surveys and studies already exist, however they aim at replacing a relational database in an application with a NoSQL database without bringing polyglot persistence into account. \textcite{Sadalage2012} is one notable exception in this aspect. In the case of Open Webslides, the NoSQL data store only complements the relational database and does not fulfill a critical function. Therefore, several constraints such as security and availability differ in interpretation from existing studies.

\chapter{Overview}
\label{ch:overview}

\section{Relational data stores}
\label{sec:relational-data-stores}

\Gls{rdbms} have been the de facto standard for over two decades to efficiently store and query information in a wide variety of native and web applications.
These data stores are based on the relational model devised by Edgar Codd \autocite{Codd1970}.
The data in the system is presented as relations: a collection of data consisting of rows and columns.
The user of the database can query and manipulate the data using relational operators.

Codd presented thirteen rules for a database in order for it to be considered a \textit{relational database} \autocite{Codd1985}.
However, many of the modern database systems do not adhere strictly to all of these rules.
More commonly, a relational database is defined as a database that exposes its information using a collection of rows and columns.

Relational database management systems provide certain guarantees during database transactions.
H\"arder and Reuter discussed recovery in transaction-oriented databases, and introduced the concept of ACID: an acronym referring to a set of transactional properties \autocite{Harder1983}.

\begin{enumerate}
  \item \textbf{Atomicity} \\ Each transaction is ``all or nothing'', either the transaction completes successfully and the data is mutated in an atomic way, or the transaction fails in its entirety and none of the data in the transaction is committed to the database.
  \item \textbf{Consistency} \\ Each transaction, when successful, only commits \textit{legal} results.
        This means that data in the transaction and subsequently in the database does not violate any constraints. The database is always in a consistent state.
  \item \textbf{Isolation} \\ Actions within a single uncommitted transaction are not visible to other, concurrently running transactions.
        Once the transaction has successfully completed, the data is visible for the other transactions.
  \item \textbf{Durability} \\ Once a transaction has completed successfully and been committed to the database, the system must guarantee that these results survive any subsequent malfunction.
\end{enumerate}

\section{NoSQL data stores}
\label{sec:nosql-data-stores}

By 2009, a totally different concept of data storage was popularized \autocite{Leavitt2018}.
NoSQL data stores provide a system of storage that is ``non SQL'' or ``non relational'' \autocite{NoSQL2018}.
Furthermore, properties of NoSQL data stores included horizontal scalability, inherently distributed and open source.
More recently, the NoSQL denomination has also been explained as ``not only SQL'', pointing on the fact that most NoSQL data stores provide a different interface besides SQL.
Many databases expose a REST API as primary execution interface.
Other data stores have devised their own binary communication protocol, such as the Bolt protocol \autocite{Bolt2015} for the Neo4j graph data store.

% MapReduce?
Another concept that became popular together with the increase in dataset size is MapReduce.
MapReduce is a conceptual programming framework and physical implementation for processing big data sets using multiple, distributed nodes \autocite{Dean2008}.
Some NoSQL data stores support this paradigm natively, while others have later added support for it.

The use of non-relational data stores was motivated by the needs of Web 2.0 companies such as Facebook and Google \autocite{Mohan2013}.
NoSQL provides a way to store and process massive amounts of data in a flexible way.
The architecture of such systems is usually more simple than the equivalent relational database systems, and are more aimed at improving horizontal scalability as opposed to vertical scalability.
Data is not stored in rows and columns as is the case in the relational model, but rather in a different data structure.

\textcite{Sadalage2012} reject the proposition that NoSQL data stores replace relational data stores.
Rather, the technology is meant to complement the relational one, and substituting one for another is not deemed to be a potential solution for performance issues.
Since both systems are designed from the ground up with very different ideas in mind, developers have to think about the potential advantages and pitfalls of each.
NoSQL has certain use cases where it shines, whereas the relational data model is a much better fit for other purposes.
The use of multiple data storage technologies and database management systems within the same application is called \textit{polyglot persistence}.

\textcite{Nayak2013} divide the data models used by NoSQL data stores into five categories.
The following sections describe these five categories, and present three additional categories that can be identified in recent NoSQL database trends.

\subsection{Key-value}
\label{subsec:key-value}

Key-value data stores are simple in design, yet powerful and efficient when used in the right circumstances.
A key-value data store allows the user to store schemaless data using an opaque, unique key, creating a \textit{key} and \textit{value} pair.
The values are stored in a manner similar to hash tables or dictionaries commonly found in programming languages and standard libraries
Queries are processed by looking up the value for the provided key, which is used as an index in the database.
Modern key-value data stores prefer high scalability over consistency, resulting in the fact that more advanced ad-hoc querying and analytical operations on the data such as joins are not available.

\subsection{Document}
\label{subsec:document}

Document databases store their data in \textit{documents}, indexed by a unique key.
Documents are usually structured in a hierarchical manner and represented in the JSON format.
Document stores are technically a subclass of key-value stores, however the difference lies in the interpretation of the data itself.
Since documents are schemaless, each document may have a similar structure, or a completely different one.
In contrast with key-value data stores, the value (document) is not opaque to the database management system, but is parsed and interpreted, and subsequently used for query optimization.
Some document data stores may provide advanced query capabilities on the contents of documents.

\subsection{Column-oriented}
\label{subsec:column-oriented}

Column-oriented data stores are designed to store data by column rather than by row.
This kind of NoSQL data store is more similar to relational database systems than the other categories, in regard to data structure used to store the data \autocite{Abadi2008}.
In column-oriented data stores, each key is associated with a set of attributes, stored in columns.
Concretely this means that the data is indexed by column value, rather than by row.
Column-oriented data stores are commonly used for queries where only a subset of the attributes is retrieved, as opposed to row-based data stores where the entire row is returned, after possibly discarding any unused values \autocite{Abadi2009}.

\subsection{Graph}
\label{subsec:graph}

Graph data stores keep their data persisted in the form of a graph.
The graph is made up of \textit{nodes} and \textit{edges} as the graph-theoretical model describes \autocite{West2001}.
The former are the database entities that contain the data itself, similar to tables and their respective columns in the relational model.
The latter are the relations between these entities, and can also contain attributes.
Graph data stores use a technique called \textit{index-free adjacency}, where every node contains a logical pointer to the adjacent node \autocite{Weinberger2016}.
This makes graph traversal a very fast operation.
Some graph databases like Neo4j are ACID compliant \autocite{miller2013}.

\subsection{Object-oriented}
\label{subsec:object-oriented}

Object-oriented data stores represent the data as an object, closely resembling the concept of an object in object-oriented programming languages.
This puts object-oriented data stores much closer to the programming environment than other database systems.
It provides all features inherent to object-oriented languages, such as data encapsulation, inheritance and polymorphism.
The concepts of class, class attributes and an object can be mapped onto the relational concepts of a table, columns and a tuple.
This concept of data storage follows the programming model much better and makes software development more flexible.

\subsection{Multi-model}
\label{subsec:multi-model}

Data stores are generally built and optimized around one data model.
However, databases supporting multiple data models exist as well.
These database systems allow storing data using any of the different data models mentioned before, while integrating these into the same server package.
One example of such a data store is OrientDB \autocite{OrientDB2010}.
Multi-model data stores support and facilitate the principle of polyglot persistence while reducing operational complexity of running multiple database management systems.

\subsection{NewSQL}
\label{subsec:newsql}

NewSQL is a type of relational database management system that aims to provide the same scalability and distributed performance of NoSQL data stores \autocite{Grolinger2013}.
NewSQL data stores are ACID compliant.

\subsection{Triple store}
\label{subsec:triple-store}

A triple store is a type of data store similar to key-value and graph data stores.
Triple stores process data using semantic queries on data triples.
A triple consists of a \textit{subject}, an \textit{predicate} and an \textit{object} \autocite{Rohloff2007}.
\\

Most NoSQL data stores are built around the concept of \textit{eventual consistency} \autocite{Brewer2000}.
This is a consistency model that dictates that all accesses to a particular piece of data will eventually return the last updated value.
This principle is broadly implemented in distributed computing systems.
Systems providing this property are also classified as BASE: Basically Available, Soft state, Eventual consistency.
In contrast to the ACID properties, systems built around the BASE principles prefer availability over consistency.

In 2000, Eric Brewer presented a conjecture known as the CAP theorem \autocite{Brewer2000}.
This conjecture, later formally proven \autocite{GilbertLynch2002}, asserts that it is impossible for a distributed data store to exhibit more than two out of three of the following properties:

\begin{enumerate}
  \item \textbf{Consistency}: Every read operation receives the most recent write result
  \item \textbf{Availability}: Every request receives a non-error result
  \item \textbf{Partition tolerance}: The system continues to work despite failure to communicate between nodes
\end{enumerate}

\begin{figure}
  \centering
  \includegraphics[width=.6\textwidth]{img/cap-theorem.png}
  \caption{CAP Theorem}
  \label{fig:cap-theorem}
\end{figure}

The CAP theorem states that when a network partition is present, the database developer has to choose between providing consistency or availability.
Note that \textit{consistency} as defined by the CAP theorem is not the same concept as consistency as described by the ACID properties.
Database systems respecting the ACID guarantees choose consistency over availability, while database systems built on the BASE principle generally choose availability over consistency.

%%=============================================================================
%% Methodologie
%%=============================================================================

\chapter{Methodology}
\label{ch:methodology}


\chapter{Data stores}
\label{ch:data-stores}

\section{Selected data stores}
\label{sec:selected-data-stores}

% TODO: reference for large number
Due to the large number of active and maintained NoSQL data stores it is not feasible to consider all of them for this research.
We will consider every NoSQL category, discuss the use of the respective categories applied to the use case, and decide whether or not it will be included in the research.
In the applicable NoSQL categories the most popular data stores are selected, where popularity is based on certain predetermined parameters.

\textcite{DBEngine2018} maintains a list of database systems ranked by popularity, based on parameters such as number of mentions on websites, Google Trends and relevance in social networks.
% TODO: reference for LinkedIn and Upwork
These parameters are also cross-referenced with professional networks such as LinkedIn and Upwork using the number of available job offers and professional profiles.
This ranking is called the DB-Engine Ranking.
The list includes not only NoSQL databases but also other types of data storage systems such as relational database systems.

% TODO: reference for Gartner
The information technology division of Gartner maintains a yearly report on emerging relational and non-relational database technologies.
% TODO: elaborate

The DB-Engine Ranking is used primarily to determine which data stores are the most interesting to consider in this research.
The Magic Quadrant for Operational Database Management Systems was used as a secondary resource.
Due to licensing concerns regarding the Open Webslides project, database systems that do not fall under a free license as classified by the \textcite{FreeSoftwareFoundation1985} will not be considered.

% TODO: references
Certain categories of NoSQL data stores will not be considered.
Key-value stores will not be accounted for due to the relative simplicity of this type of data store.
The main use case of key/value stores is not storing more complex information, rather the emphasis lies more on scalability and consistency.
Processing complex queries that consist of the NoSQL equivalent of relational joins is not efficient in key-value data stores.
Implementing the proposed data model and queries in such a system is a more ambitious task that does not lie within the scope of the research.

Column-oriented data stores are a category of NoSQL data stores that is questionably useful in this case study.
% TODO: reference
These data stores are commonly seen as inverse relational database systems, where the storage of attributes per entity is more flexible regarding nullable values and unstructured information.
Column-oriented data stores are very efficient when retrieving a subset of columns for a certain record.
Since the data store will be deployed as additional data store next to the authoritative relational database, it will not contain any data that will not be used when querying the database.
This workflow cancels out the efficiency and usefulness of column-oriented data stores, and subsequently we will not consider this NoSQL category as a viable candidate.

Furthermore, comparison and application of NewSQL data stores will not be included in this paper either.
As described in \cref{ch:overview}, NewSQL aims to provide scalable performance similar to that of NoSQL while still guaranteeing ACID properties.
Since ACID is not a major concern for this use case, NewSQL does not provide significant advantage over NoSQL, and will therefore be omitted from the comparison.
Similarly, object-oriented and multi-model database such as OrientDB \autocite{OrientDB2010} are not within scope of this research.

In conclusion, document and graph data stores will provide the most beneficial data storage and are the main focus of this research.\\

The most popular document database systems according to the DB-Engine Ranking are MongoDB \autocite{MongoDB2009}, CouchDB \autocite{CouchDB2005} and Couchbase \autocite{Couchbase2010}.
While Couchbase is technically a multi-model data store, it is being ranked as a document database.
However, Couchbase is a conceptual merge between CouchDB and Membase, and mainly adds improved scalability, clustering and auto-sharding.
Couchbase will not be investigated upon in a practical capacity, however it is considered as an option to increase multi-host scalability.
This leads to MongoDB and CouchDB being the contestants in the document database system category.

Finally, as sole graph database the Neo4j system \autocite{Neo4j2007} stands out greatly over competitors in the ranking.
This database system will be analyzed as only graph database in this research.

In conclusion, the data stores included in the study will be MongoDB and CouchDB as document databases, and Neo4j as graph databases.

\section{Comparative study}
\label{sec:comparative-study}

Data stores and databases can be compared and analyzed using many quantitative and qualitative aspects.
In this section we propose a selection of criteria based on the usefulness applied to the studied use case.
Since the landscape and feature-set of NoSQL data stores is changing on a weekly base, the impact of these technologies must be carefully considered in order to reach a durable conclusion in this research.

The features of a data store that are taken into account in this comparative study are:

\begin{itemize}
  %% Features
  \item Querying capabilities
    \begin{itemize}
      \item Language
      \item Protocol
      \item MapReduce
    \end{itemize}

  %% Language
  \item Programming language
  \item Language bindings

  %% Integrity
  \item Integrity model
  \item Atomicity
  \item Revision control
  \item Consistency

  %% Scalability
  \item Persistence
  \item Partitioning
  \item High Availability
  \item Concurrency
  \item Replication

  %% Various
  \item License
  \item Commercial support
\end{itemize}

Certain aspects are not relevant to the presented use case because of various reasons.
Aspects omitted from the study are:

\begin{itemize}
  \item Security
  \item Compression
  \item Full-text search
  \item Geospatial functionality
  \item Cloud hosting
\end{itemize}

%% Omitted aspects
The specific use case described in this thesis does not focus on security, since the data store is not public facing and clients do not interact directly with it.
Security measures include but are not limited to authentication, authorization, encryption and auditing.
None of these are features that are required or useful for the comparison.
Authentication and authorization is functionality that is present in all databases.
It introduces the concept of multiple clients or roles connecting to a database, and assigning permissions to these clients in order to enforce permissions-based access.
However since there will only be one client connecting to the database - the platform itself - deeper integration with LDAP, ActiveDirectory or similar is superfluous and omitted from the comparison.

Encryption refers to the mechanism where data is encrypted and unreadable for unauthorized third parties.
Encryption in databases is threefold: encryption of data at rest, client-to-server communication and server-to-server communication \autocite{Grolinger2013}.
However, since data protection is a comprehensive topic, it does not fall within the scope of this research.
Consequently, encryption functionality is not considered in the comparative study.

Database auditing is a facility offered by the database management system that keeps track of the usage of database resources and authorization.
Operations on the database leave a trail of events, called an \textit{audit log}.
Similarly to authentication, the usefulness of this functionality is somewhat lost when there is only one client operating on the database.
However, many security standards such as PCI-DSS and HIPAA require the existance of an audit log.

Compression of data in the database is not included in the comparison.
Builtin compression may provide additional storage space but as it is a disk space-CPU usage tradeoff we have chosen to only consider CPU usage.
Akin to compression, we leave the choice of cloud hosting up to the database administrator.

Full-text search and geospatial functions are provided by certain data stores, however these features are not used by the platform and subsequently are not relevant to this comparative study.\\

%% Comparative study

\begin{table}
  \sffamily
  \begin{tabular}{l l l l l l l l l}
    \toprule
    &
    &
    \multicolumn{3}{l}{\textbf{Querying}} &
    \multicolumn{2}{l}{\textbf{Language}} &
    & \\

    \cline{3-8}

    &
    &
    \textbf{Language} &
    \textbf{Protocols} &
    \textbf{MapReduce} &
    \textbf{Language} &
    \textbf{Language bindings} &
    \textbf{License} &
    \textbf{Commercial support} \\

    \cline{3-8}

    \multirow{2}{*}{\makecell{\textbf{Document}\\\textbf{stores}}} &
    \textbf{MongoDB} &
    JavaScript &
    \makecell{MongoDB Wire Protocol\\REST (3rd party)} &
    Yes &
    C++ &
    &
    \makecell{GNU AGPLv3 (database)\\Apache (language drivers)} &
    Yes \\

    \cline{2-8}

    &
    \textbf{CouchDB} &
    REST &
    HTTP &
    Yes &
    Erlang &
    &
    Apache &
    Yes \\

    \cline{2-8}

    \makecell{\textbf{Graph}\\\textbf{stores}} &
    \textbf{Neo4j} &
    \makecell{Cypher\\SparQL\\Gremlin (3rd party)} &
    HTTP, Bolt &
    No &
    Java &
    \makecell{.NET\\Java\\JavaScript\\Python\\Ruby (unofficially)} &
    GNU GPLv3 (Community Edition), AGPLv3 (Enterprise Edition) &
    Yes \\

    \bottomrule
  \end{tabular}

  \caption{Querying and features}
  \label{tbl:query-features}
\end{table}

\chapter{Data model}
\label{ch:data-model}

In this chapter we describe the conceptual domain of the use case, and provide logical data models applied to the NoSQL data models selected in the previous chapter.
Physical data models for MongoDB, CouchDB and Neo4j will be presented, along with an implementation in a demo Ruby on Rails application using Ruby language bindings.
Finally, a set of reference queries that may typically be used in the context of the Recent Activity feed will be proposed.
These queries will be formally described, and subsequently implemented in the query languages specific to the three data stores.

\section{Domain description}
\label{sec:domain-description}

On the Open Webslides platform, no distinction is made between a teacher and a student in the data model.
Both are represented by the \texttt{User} entity in the database.
This entity contains information pertaining to the user, such as email address, first name and last name.
The data models described in the next sections will closely follow the existing data model of the platform.
However, attributes irrelevant to the Recent Activity feed are omitted from the model and not available in the NoSQL data store, in order to improve efficiency and simplify abstraction.

A user can create or modify course content in an interactive online editor.
The actual course content, formally called a topic, is stored inside a git repository on the filesystem.
However, the platform also maintains a record of topic metadata in the relational database.
This metadata includes title and description, but also permissions and contributors on the course content.

From a technical perspective, the user has three distinct paths of action for creation and co-creation on course content.
Since the permission model in the platform is not relevant to this research, we will not go into detail on it.
First, the user can directly modify the course content if the user has permission to perform this action.
The second option is creating annotations on the topic.
This allows the user to attach private or public notes to specific content on the topic.
Annotations are stored in the relational database, in the \texttt{Annotation} entity.
The entity contains a logical pointer to the annotated content.

The final possibility to integrate user content into a topic is by adding comments.
In contrast to annotations, comments have a typical structure.
They can take the form of questions, notes, suggestions, and can also be nested - which allows simple interaction and conversation between multiple users, and effectively enables dialogue between students and teachers.

The intention of this thesis is to use the NoSQL data storage as storage mechanism for the Recent Activity feed.
This entails that the authoritative information will not be stored in that data store, but rather be extracted from the relational database whenever an activity event is generated.
Accordingly, some information may be omitted from this data store, while other information is copied.

Consider the following domain description structured as activity events in the Recent Activity feed.
These events are items that a user may typically encounter in the web application as part of the feed.

``
\textbf{\underline{John}} created \textbf{\underline{Topic A}}.
''

``
\textbf{\underline{Jane}} commented on \textbf{\underline{Topic B}}:
\textit{This is not a good example. Try and find a better one.}
''

``
\textbf{\underline{John}} commented on \textbf{\underline{Jane}'s comment} on \underline{Topic B}:
\textit{I agree.}
''

``
\textbf{\underline{Jane}} annotated \textbf{\underline{Topic A}}.
''

``
\textbf{\underline{Bob}} updated \textbf{\underline{Topic B}}.
''

``
\textbf{\underline{Bob}} reacted to \textbf{\underline{John}'s comment} on \underline{Topic B}.
''

From this description we can already derive some requirements to take into account when designing the data models.
Every event has a structure characterised by three aspects.
First, there is a user at the base of the action, and in the descriptions this is the subject of every sentence.
Second, the subject performs an action, and the actions are limited to a certain subset as determined by the developer.
Third, the user operates on an object, which is usually but not always a topic.

In the activity events, the underlined text fragments represent hyperlinks in the web application to the relevant entities.
A hyperlink for a user may link to the profile page of the user, or the contributions of the respective user on the relevant topic.
Similarly, a hyperlink for a topic may link to a page that presents an overview of the topic, or directly to course content inside the topic.
The actual destination is up to the developers of the platform, and is not directly relevant for this research.
However, the existence of these hyperlinks entails that every aspect previously described has to consist of at least one attribute that is used in the hyperlinks -- most likely this will be the unique identifier of the entity in question.\\

We present the following conventions and rules to be followed in all NoSQL data models.

\begin{itemize}
  \item The user of an activity is called the \textit{subject}
  \item The action of an activity is called the \textit{predicate}
  \item The predicate can be one of the following values:\\ \texttt{created}, \texttt{updated}, \texttt{renamed}, \texttt{commented\_on}, \texttt{annotated}, \texttt{reacted\_to}
  \item The object to which an action refers, is called the \textit{item}
  \item The item can reference topics and comments
  \item No additional attribute to facilitate hyperlinks will be included
\end{itemize}

Furthermore, since the data in the NoSQL data store is generated in function of the business-critical data in the platform, it is expected to be written to the database only once, and read many times.
This enables us to design data models where read performance is prioritized over write performance.
It is also important to note that the data is always queried in a reverse chronological way, since the Recent Activity feed displays the most recent events first.

Using this logical description of the domain, we can start to derive physical data models for the selected NoSQL data stores.

\section{Physical data model}
\label{sec:physical-data-model}

\subsection{Language bindings}
\label{subsec:language-bindings}

\Cref{tbl:query-language} lists for every NoSQL data store the programming languages for which language bindings are available.
Some of these are developed, maintained and officially supported by the database vendor, while others blossomed forth from a community effort.
For developing the physical data model, we have selected the most popular language bindings for Ruby and Ruby on Rails' ActiveModel according to RubyGems \autocite{RubyGems2003}.
If there is an official library available, it is preferred over a community library, with a view on the maintainability of the application.

For MongoDB, an officially supported language binding is available for plain Ruby and Ruby on Rails.
The latter library, called Mongoid, integrates MongoDB directly into the Rails ecosystem and was chosen as a viable candidate for developing the physical data model and queries \autocite{Mongoid2009}.

CouchDB on the contrary, does not have an officially supported Ruby binding.
The most popular gem on RubyGems is CouchRest and its ActiveModel equivalent, CouchRest Model.
However, as of the time of writing, CouchRest Model is not well maintained, with only five beta releases and no stable releases during the past three years.
Since this library provides all the necessary integrations to develop the CouchDB data model and queries, it was chosen as framework in which to implement these.

Finally, Neo4j does not have an officially supported library, however the database vendor recommends using a community supported alternative called Neo4j.rb.
This library integrates the Neo4j graph database with the Ruby on Rails stack and was subsequently used as a platform to develop the graph data model and queries.


\subsection{Document-oriented data model}
\label{subsec:document-data-model}

The fundamental building block of a document data store is a document.
The document data model provides two distinct approaches to link between different documents.
Each option has its own advantages and disadvantages, and the proposed data models attempt to use the most efficient option for the use case, despite making some trade-offs.

\begin{enumerate}
  \item \textbf{Embedded collections} (denormalized data).
        Embedding of data stores information in a single document.
        This technique is commonly used when the entity \textit{contains} the embedded entity: for example storing contact details of a user.
        Another use for this approach is storing one-to-many relationships, where the child documents are always queried within the context of the parent document.
  \item \textbf{References} (normalized data): Storing a reference to another document, similar to storing keys to other tables in the relational model.
\end{enumerate}

Embedded collections provide better performance, since the embedded document is included in the parent document and the database management system does not have to execute an additional query.
The disadvantage of embedding is that data may be duplicated, if an embedded document is included in multiple parent documents.
Using referenced documents yields the exact opposite effects: slower performance due to additional queries, yet less data duplication in case of a multiply referenced document.

Since the data model has to be optimized for read performance, we will try to use embedded documents wherever possible.

From the domain description of the use case, we can identify one main entity which may be stored in its own collection: \texttt{Event}.
This is the entrypoint of the Recent Activity feed, and the event document will embed or reference all other entities.
The first relation that can be identified is the subject relation: every event has exactly one subject.
However, we can infer that \texttt{Event} is not the only entity that has a link to \texttt{Subject}.
Every comment also references \texttt{Subject}.
Since the only information included in the \texttt{Subject} entity is a name, we have chosen to embed \texttt{Subject} in the parent documents.
The data duplication of a single attribute is negligible as opposed tot the performance penalty encountered when using a referenced document in this case.

The same train of thought can be applied to \texttt{Topic}: both \texttt{Event} and \texttt{Comment} reference the entity, yet it only contains one attribute.
Subsequently \texttt{Topic} will be used only as an embedded document as well.

One downside of this approach is that when a user changes the name or title of a subject or topic respectively, the existing data in the NoSQL database does not get updated and the Recent Activity feed may display outdated information.

The way a comment gets stored in the document data store is very particular.
The storage of both a comment and an event referencing the comment in different collections is redundant, since the database will never be queried from the perspective of a comment.
Considering this, every top-level comment -- comments made on a topic -- is stored as a \texttt{text} attribute of an event.
This way the \texttt{item} of that event still refers to the topic itself.
Events for child comments -- comments made on another comment -- are structured differently.
In this case, the \texttt{item} refers to the parent comment, but does not include its text.
This flexible approach allows us to store the information efficiently.

This leads us to the following physical document data models, implemented in the Ruby on Rails application.

\subsubsection{MongoDB implementation}
\label{subsubsec:mongodb-implementation}

The physical data model for MongoDB is implemented using the Mongoid library, which provides Ruby and Ruby on Rails bindings to the data store.
Mongoid is officially supported by Mongo, Inc.

The models developed during this research are available in \cref{ch:source-code}.

\subsubsection{CouchDB implementation}
\label{subsubsec:couchdb-implementation}

An attempt was made to provide a physical implementation of the proposed document data model using the CouchRest Model library \autocite{Couchrest2011}.
CouchRest Model provides Ruby on Rails integrations and is built on the CouchRest Ruby library.

During the development of the data models in question, many roadblocks were encountered that impeded development or even made it impossible to continue.
The library does not support object inheritance, and relationship polymorphism for instance.

The degree in which CouchDB is compatible with Ruby on Rails was deemed not satisfactory, and as such CouchDB was dropped for the physical implementation of the queries.

The models developed during this research are available in \cref{ch:source-code}.

\subsection{Graph-oriented data model}
\label{subsec:graph-data-model}

Using the examples of activity events in \cref{sec:domain-description}, several entities can be identified.
These entities are used to model the \textit{nodes}, \textit{labels} and \textit{edges} in the graph data stores

The first step is to extract the nodes from the description.
In the domain, there are four main entities.

\begin{itemize}
  \item Event
  \item Subject
  \item Topic
  \item Comment
\end{itemize}

Similarly to the document data model, \texttt{Event} represents an entry in the Recent Activity feed.

Neo4j data modeling also supports labels, a graph construct that groups nodes into sets.
A set contains all nodes that are labeled with the same label.
A node can have any number of labels.

In the use case, there is one important opportunity to make use of node labels.
The \texttt{item} relation of \texttt{Event} can reference multiple other entities, in this case \texttt{Topic} and \texttt{Comment}.
Using the label \texttt{Item} on top of the \texttt{Topic} and \texttt{Comment} labels allows room for future expansion to other entity types.

A number of relationships can be identified in the domain description.

\begin{itemize}
  \item \texttt{Event} has one \texttt{Subject}
  \item \texttt{Event} has one \texttt{Item}
  \item \texttt{Comment} has one \texttt{Subject}
  \item \texttt{Comment} has one \texttt{Topic}
\end{itemize}

These relationships are modeled in the graph data store as edges.

This leads to the graph data model in figure \ref{fig:graph-model}.

\begin{figure}
  \centering
  \includegraphics[width=.8\textwidth]{img/graph-model.png}
  \caption{Graph data model}
  \label{fig:graph-model}
\end{figure}

\subsubsection{Neo4j implementation}
\label{subsubsec:neo4j-implementation}

The physical data model for Neo4j is implemented using the community-supported Neo4j.rb library, which provides Ruby and Ruby on Rails bindings to the data store \autocite{Neo4jrb2010}.

The models developed during this research are available in \cref{ch:source-code}.

\section{Reference queries}
\label{sec:reference-queries}

In order to perform an empricial analysis on the selected data stores and the proposed physical data models, we present five reference queries in this chapter.
These reference queries will reflect the method of querying that would be the most common in the physical platform implementation, and mirrors the way data is queried from a database perspective.
All reference queries will be implemented using the available language bindings, however the generated implementation-specific query will also be presented.

Since the data of the use case is aimed at a write-once, read-many character, the majority of queries will not touch the data itself, but rather only read it.
Four read-only queries are included in the following sections, and one query that will insert new data into the data store.

\subsection{Querying}
\label{subsec:querying}

\subsubsection{Query 1}
\label{subsubsec:query-1}

``
Select N most recent events, ordered reverse chronologically
''

This query, as most simple reference query presented, is an example of a query that can be used on the homepage of the platform.
When a user opens the web application, a reverse chronologically ordered list of events is presented.
This view allows for a quick overview of the activity in the platform, and since the user is not signed in yet, it is not tailored.
This also means that everyone who visits the platform without signing in will receive the same events in their Recent Activity feed.

\subsubsection*{MongoDB}

\begin{listing}[H]
  \begin{minted}[baselinestretch=1,fontsize=\footnotesize]{ruby}
  MongoDB::Event
    .all
    .order_by(:created_at => :desc)
    .limit(count)
    .each(&:to_s)
  \end{minted}

  \caption{MongoDB query 1}
  \label{lst:mongodb-query-1}
\end{listing}

\subsubsection*{Neo4j}

\begin{listing}[H]
  \begin{minted}[baselinestretch=1,fontsize=\footnotesize]{ruby}
  Neo4j::Event
    .all
    .order(:created_at => :desc)
    .limit(count)
    .each(&:to_s)
  \end{minted}

  \caption{Neo4j query 1}
  \label{lst:neo4j-query-1}
\end{listing}

\begin{listing}
  \begin{minted}[baselinestretch=1,fontsize=\footnotesize]{cypher}
  MATCH (result_neo4jevent:`Event`)
    RETURN result_neo4jevent
    ORDER BY result_neo4jevent.created_at DESC
    LIMIT {limit_1} | {:limit_1=>count}

  MATCH (previous:`Event`)
    WHERE (ID(previous) = {ID_previous})
    OPTIONAL MATCH (previous)-[rel1:`by`]->(next:`Subject`)
    RETURN
      ID(previous),
      collect(next) | {:ID_previous=>result_neo4jevent.id}

  MATCH (previous:`Event`)
    WHERE (ID(previous) = {ID_previous})
    OPTIONAL MATCH (previous)-[rel1:`on`]->(next:`Item`)
    RETURN
      ID(previous),
      collect(next) | {:ID_previous=>result_neo4jevent.id}
  \end{minted}

  \caption{Neo4j query 1 (CYPHER)}
  \label{lst:neo4j-query-1-cypher}
\end{listing}

The Neo4j ORM call get converted into three distinct queries: one to get the \texttt{Event} node and two queries to get the related nodes \texttt{Subject} and \texttt{Item}.

\subsubsection{Query 2}
\label{subsubsec:query-2}

``
Select N most recent events, where the event is related to a topic in a list of given topics, ordered reverse chronologically
''

A user has to ability to subscribe to topics, which means that the Recent Activity feed may be tailored to the user.
Once the user logs in to the platform, the Recent Activity feed can be presented in a more attractive way.
The events in the feed will then consist of only events related to topics the user has subscribed to (which also includes the topics where the user is author or contributor).

\subsubsection*{MongoDB}

\begin{listing}[H]
  \begin{minted}[baselinestretch=1,fontsize=\footnotesize]{ruby}
  # List of subscribed topic identifiers
  topic_ids = [...]

  MongoDB::Event
    .in('item._id' => topic_ds)
    .order_by(:created_at => :desc)
    .limit(count)
    .each(&:to_s)
  \end{minted}

  \caption{MongoDB query 2}
  \label{lst:mongodb-query-2}
\end{listing}

\subsubsection*{Neo4j}

\begin{listing}[H]
  \begin{minted}[baselinestretch=1,fontsize=\footnotesize]{ruby}
  # List of subscripted topic identifiers
  topic_ids = [...]

  Neo4j::Topic
    .where(:id => topic_ids)
    .events
    .order_by(:created_at => :desc)
    .limit(count)
    .each(&:to_s)
  \end{minted}

  \caption{Neo4j query 2}
  \label{lst:neo4j-query-2}
\end{listing}

\begin{listing}[H]
  \begin{minted}[baselinestretch=1,fontsize=\footnotesize]{cypher}
  MATCH (node2:`Topic`:`Item`)
    WHERE (node2.uuid IN {node2_uuid})
    MATCH (node2)<-[rel1:`on`]-(result_events:`Event`)
    RETURN result_events
    ORDER BY result_events.created_at DESC
    LIMIT {limit_1} | {:limit_1=>1, :node2_uuid=>topic_ids}

  MATCH (previous:`Event`)
    WHERE (ID(previous) = {ID_previous})
    OPTIONAL MATCH (previous)-[rel1:`by`]->(next:`Subject`)
    RETURN
      ID(previous),
      collect(next) | {:ID_previous=>node2.id}

  MATCH (previous:`Event`)
    WHERE (ID(previous) = {ID_previous})
    OPTIONAL MATCH (previous)-[rel1:`on`]->(next:`Item`)
    RETURN
      ID(previous),
      collect(next) | {:ID_previous=>node2.id}
  \end{minted}

  \caption{Neo4j query 2 (CYPHER)}
  \label{lst:neo4j-query-2-cypher}
\end{listing}

\subsubsection{Query 3}
\label{subsubsec:query-3}

``
Select N most recent events, where the event is related to a given topic, ordered reverse chronologically
''

Every topic also has an overview page, which mainly contains metadata such as description, author, contributors and other information not directly related to the course content.
Next to the metadata, a custom Recent Activity feed is also included on the page.
This feed only contains events related to the topic the user is currently viewing, and effectively presents a timeline of changes and discussions.

\subsubsection*{MongoDB}

\begin{listing}[H]
  \begin{minted}[baselinestretch=1,fontsize=\footnotesize]{ruby}
  # Topic identifier
  topic_id = ...

  MongoDB::Event
    .where('item._id' => topic_id)
    .order_by(:created_at => :desc)
    .limit(count)
    .each(&:to_s)
  \end{minted}

  \caption{MongoDB query 3}
  \label{lst:mongodb-query-3}
\end{listing}

\subsubsection*{Neo4j}

\begin{listing}[H]
  \begin{minted}[baselinestretch=1,fontsize=\footnotesize]{ruby}
  # Topic identifier
  topic_id = ...

  Neo4j::Topic
    .find(topic_id)
    .events
    .order(:created_at => :desc)
    .limit(count)
    .each(&:to_s)
  \end{minted}


  \caption{Neo4j query 3}
  \label{lst:neo4j-query-3}
\end{listing}

\begin{listing}[H]
  \begin{minted}[baselinestretch=1,fontsize=\footnotesize]{cypher}
  MATCH (neo4j_topic)
    WHERE (ID(neo4j_topic) = {ID_neo4j_topic})
    MATCH (neo4j_topic)<-[rel1:`on`]-(result_events:`Event`)
    RETURN result_events
    ORDER BY result_events.created_at DESC
    LIMIT {limit_1} | {:limit_1=>1, :ID_neo4j_topic=>topic_id}

  MATCH (previous:`Event`)
    WHERE (ID(previous) = {ID_previous})
    OPTIONAL MATCH (previous)-[rel1:`by`]->(next:`Subject`)
    RETURN
      ID(previous),
      collect(next) | {:ID_previous=>neo4j_topic.id}

  MATCH (previous:`Event`)
    WHERE (ID(previous) = {ID_previous})
    OPTIONAL MATCH (previous)-[rel1:`on`]->(next:`Item`)
    RETURN
      ID(previous),
      collect(next) | {:ID_previous=>neo4j_topic.id}
  \end{minted}

  \caption{Neo4j query 3 (CYPHER)}
  \label{lst:neo4j-query-3 (CYPHER)}
\end{listing}

\subsubsection{Query 4}
\label{subsubsec:query-4}

``
Select N most recent events, where the event is related to a given user, ordered reverse chronologically
''

Similarly to topics, a profile page also includes a timeline of the user's activities: additions, deletions, comments and annotations that were recently made by that user.

\subsubsection*{MongoDB}

\begin{listing}[H]
  \begin{minted}[baselinestretch=1,fontsize=\footnotesize]{ruby}
  # Subject identifier
  subject_id = ...

  MongoDB::Event
    .where('subject._id' => subject_id)
    .order_by(:created_at => :desc)
    .limit(count)
    .each(&:to_s)
  \end{minted}

  \caption{MongoDB query 4}
  \label{lst:mongodb-query-4}
\end{listing}

\subsubsection*{Neo4j}

\begin{listing}[H]
  \begin{minted}[baselinestretch=1,fontsize=\footnotesize]{ruby}
  # Subject identifier
  subject_id = ...

  Neo4j::Subject
    .find(subject_id)
    .events
    .order(:created_at => :desc)
    .limit(count)
    .each(&:to_s)
  \end{minted}

  \caption{Neo4j query 4}
  \label{lst:neo4j-query-4}
\end{listing}

\begin{listing}[H]
  \begin{minted}[baselinestretch=1,fontsize=\footnotesize]{cypher}
  MATCH (n:`Subject`)
    WHERE (n.uuid = {n_uuid})
    RETURN n
    ORDER BY n.uuid
    LIMIT {limit_1} | {:n_uuid=>subject_id, :limit_1=>1}

  MATCH (neo4j_subject)
    WHERE (ID(neo4j_subject) = {ID_neo4j_subject})
    MATCH (neo4j_subject)<-[rel1:`by`]-(result_events:`Event`)
    RETURN result_events
    ORDER BY result_events.created_at DESC
    LIMIT {limit_1} | {:limit_1=>1, :ID_neo4j_subject=>subject_id}

  MATCH (previous:`Event`)
    WHERE (ID(previous) = {ID_previous})
    OPTIONAL MATCH (previous)-[rel1:`by`]->(next:`Subject`)
    RETURN
      ID(previous),
      collect(next) | {:ID_previous=>result_events.id}

  MATCH (previous:`Event`)
    WHERE (ID(previous) = {ID_previous})
    OPTIONAL MATCH (previous)-[rel1:`on`]->(next:`Item`)
    RETURN
      ID(previous),
      collect(next) | {:ID_previous=>result_events.id}
  \end{minted}

  \caption{Neo4j query 4 (CYPHER)}
  \label{lst:neo4j-query-4}
\end{listing}

\subsection{Insertion}
\label{subsec:insertion}

\subsubsection{Query 5}
\label{subsubsec:query-5}

``
Insert N given :created or :updated events
''

Finally, since read requests will most likely outnumber write requests with several magnitudes in the studied use case, only one query where data is inserted is presented.
The query creates a single event in the data store.

\subsubsection*{MongoDB}

\begin{listing}[H]
  \begin{minted}[baselinestretch=1,fontsize=\footnotesize]{ruby}
  # Subject identifier
  subject_id = ...

  # Item identifier
  item_id = ...

  MongoDB::Event.create! :subject => subject_id,
                         :item => item_id,
                         :predicate => :updated
  \end{minted}

  \caption{MongoDB query 5}
  \label{lst:mongodb-query-5}
\end{listing}

\subsubsection*{Neo4j}

\begin{listing}[H]
  \begin{minted}[baselinestretch=1,fontsize=\footnotesize]{ruby}
  # Subject identifier
  subject_id = ...

  # Item identifier
  item_id = ...

  Neo4j::Event .create! :subject => subject_id,
                        :item => topic_id,
                        :predicate => :updated
  \end{minted}

  \caption{Neo4j query 5}
  \label{lst:neo4j-query-5}
\end{listing}

\begin{listing}[H]
  \begin{minted}[baselinestretch=1,fontsize=\footnotesize]{cypher}
  MATCH (n:`Subject`)
    WHERE (n.uuid = {n_uuid})
    RETURN n
    ORDER BY n.uuid
    LIMIT {limit_1} | {:n_uuid=>subject_id, :limit_1=>1}

  MATCH (n:`Item`)
    WHERE (n.uuid = {n_uuid})
    RETURN n
    ORDER BY n.uuid
    LIMIT {limit_1} | {:n_uuid=>item_id, :limit_1=>1}

  CREATE (n:`Event`)
    SET n = {props}
    RETURN n | {:props=>{:uuid=>event_id, :created_at=>1526737892, :predicate=>1}}

  MATCH (n:`Subject`)
    WHERE (n.uuid = {n_uuid})
    RETURN n
    LIMIT {limit_1} | {:n_uuid=>subject_id, :limit_1=>1}

  MATCH
      (from_node),
      (to_node)
    WHERE
      (ID(from_node) = {ID_from_node}) AND
      (ID(to_node) = {ID_to_node})
    CREATE (from_node)-[rel:`by` {rel_create_props}]->(to_node)
      | {:ID_from_node=>event_id, :ID_to_node=>subject_id, :rel_create_props=>{}}

  MATCH (n:`Item`)
    WHERE (n.uuid = {n_uuid})
    RETURN n
    LIMIT {limit_1} | {:n_uuid=>item_id, :limit_1=>1}

  MATCH
      (from_node),
      (to_node)
    WHERE
      (ID(from_node) = {ID_from_node}) AND
      (ID(to_node) = {ID_to_node})
    CREATE (from_node)-[rel:`on` {rel_create_props}]->(to_node)
      | {:ID_from_node=>event_id, :ID_to_node=>item_id, :rel_create_props=>{}}
  \end{minted}

  \caption{Neo4j query 5 (CYPHER)}
  \label{lst:neo4j-query-5-cypher}
\end{listing}

\clearpage{}

\section{Conclusion}
\label{sec:data-model-conclusion}

In this chapter we presented an introduction to the domain, and provided some examples of events in the Recent Activity feed.
Furthermore, we analyzed this domain description, and derived a logical data model for both document- and graph-oriented data stores.
Next, we proposed an implementation of this logical data model for one document-oriented, and one graph-oriented data store in the Ruby language bindings available for the respective database management systems.
Finally, we presented five reference queries for the three data stores, and included both a language binding-specific DSL and a physical query implementation for each.

The CouchDB implementation was dropped due to the most popular Ruby language binding available not being up to date and lacking many essential features.

\chapter{Empirical study}
\label{ch:empirical-study}

In this chapter the previously proposed physical data models for MongoDB and Neo4j are implemented and put to the test.
A Ruby on Rails application was developed that uses Mongoid and Neo4j.rb libraries to provide connectivity to the data stores.
Custom benchmarking scripts were implemented using the RSpec library \autocite{RSpec2005} and the \texttt{Benchmark} class built into the Ruby implementation.

\section{Previous work}
\label{sec:previous-work}

% Netflix benchmark, http://www.bgbenchmark.org/BG/, XGDBench

In current literature there are already some studies present that compare NoSQL data stores based on a performance review.
\textcite{Abramova2014} compare the Cassandra, HBase, MongoDB, OrientDB and Redis data stores.
The authors are using the Yahoo! Cloud Serving Benchmark \autocite{Cooper2010}, which presents a framework to facilitate performance comparisons for cloud-based systems by providing a core set of benchmarks.
The paper concludes that Redis, as in-memory database, provides the best performance in query processing.
Redis is optimized for \textit{get} and \textit{put} operations due to in-memory data mapping.
On the other hand, Cassandra and HBase are optimized for update operations.

Finally, MongoDB was found to be the data store with the slowest execution times, having an overall performance that was more than 58 times lower in comparison with Redis.
This proves that in-memory mapping of data results in a very performant query processing system.

\textcite{Schmid2015} develop a performance comparison aimed at applications using geo-functionalities present in the database management system.
The authors conclude that requests purely on attributes NoSQL data stores are superior over relational data stores.
For requests using geo-functions the NoSQL data stores also perform constant for increasing dataset sizes.
For smaller datasets with a more interlinked architecture, the relational data stores perform predictably better.

\textcite{Barahmand2015} quantify the horizontal and vertical scalability of MongoDB and HBase in the context of a social benchmarking framework named BG \autocite{Barahmand2013}.
This benchmarking framework models a social graph in the data store and performs simple operations reading and writing small amounts of data with in a social interaction context.
The researchers found that both data stores scale superlinearly, limited by the complete utilization of certain nodes in the cluster.

The experimental comparison by \textcite{Kolomivcenko2013} concludes that Neo4j is the most performant product under the compared graph data stores, especially in graph traversal queries.
The authors also indicate that MongoDB performs well in queries that are not or lightly graph related.

During the research of this thesis, the decision not to use BG or the Yahoo! Cloud Serving Benchmark was made.
First, there is already literature in the field concerning these benchmarks and the data stores that were selected for comparison in this thesis.
These sources provide additional input when formulating an answer on the research questions.
However, developing custom benchmarks adjusted to the workload and environment the data schema is intended to be used in, present a more realistic view of real-word performance.
This is coupled with the fact that this research delivers a Ruby implementation ready to be integrated into the existing platform.

\section{Experimental setup}
\label{sec:experimental-setup}

In order to keep the results of the tests consistent, the following rules are applied:

\begin{itemize}
  \item Query caching is disabled in Mongoid. Neo4j.rb does not have an equivalent feature.
  \item Connection pooling is disabled
  \item Clustering is disabled as horizontal scalability performance is not within the scope of this research
  \item No additional performance tweaks were applied on the database management systems
\end{itemize}

Mongo Wire Protocol and Bolt Protocol were chosen as native protocols for MongoDB and Neo4j respectively.

All tests were performed on a single machine with the following specifications.

\begin{itemize}
  \item Ruby 2.5.0, operating under Arch Linux
  \item Intel Core i7-3840QM (4 cores, 8 threads)
  \item Hyperthreading and \TODO{Intel Turbo Boost} enabled
  \item 32GB DDR4 RAM
  \item 180GB SATA-III SSD
\end{itemize}

The data store versions that were tested are the following versions.

\begin{itemize}
  \item MongoDB 3.6.3
  \item Neo4j Community Edition 3.0.6
\end{itemize}

\section{Procedure}
\label{sec:procedure}

The following procedure was followed throughout the performance benchmarks.

First, the database was filled with random testing data.
The script in \TODO{listing whatever} was developed in order to create random data and insert this into each of the data stores.
The variable \texttt{FACTOR} in the script is a multiplication factor that directly influences the amount of data generated.

Next, every test was ran sequentially for the data stores, and the timing results were written to separate files.
During the tests all database management systems were running in the background.

Since the execution time of a single iteration of a query is negligable, every query was executed a number of times to negate the effect of external factors, such as operating system scheduling and I/O wait times.
The effects of varying the iteration count are discussed in \cref{subsec:query-size}.
Having multiple iteration runs allows analyzing the vertical scalability of the query in limited fashion as well.
As indicated in the informal descriptions of the reference queries, the query itself is also dependent on a variable $N$ which signifies the event count that is to be retrieved from the database.
The effects of varying query sizes were analyzed and discussed in \cref{subsec:query-size}.
A sane default for query size in a concrete implementation could be 100, meaning that 100 events are retrieved every time the user loads the Recent Activity page.

In summary, there are three variables that can be modified in the performance tests:

\begin{itemize}
  \item \textbf{Data multiplication factor}: total size of the dataset in the database
  \item \textbf{Iteration count}: number of sequentially ran iterations of the query
  \item \textbf{Query size}: number of retrieved events from the database
\end{itemize}

\section{Results}
\label{sec:results}

Small-scale, informal tests determined that the Neo4j implementation is many times slower than the MongoDB implementation.
Hence, the the benchmarks are ran independently for MongoDB and Neo4j, using an iteration count that is magnitudes smaller for Neo4j, yet yielding roughly the same execution times.

The execution times represent the real wall-clock time elapsed for every query.
It is measured from the moment the query gets dispatched to the ORM, and hence includes the time to traverse the entire software stack, including instantiation of the database objects in Ruby.
This overhead is intended to be included in the measurements, since the total time a user has to wait for a database query is influenced both by the software stack and the database management system.

\subsection{Dataset size}
\label{subsec:dataset-size}

First, the effects of the dataset size on the query performance are analyzed.
By varying the multiplication factor in the seed data generator the size of dataset can be controlled.

For MongoDB, an iteration count of 50 000 was used, while Neo4j was provided only 1000 iterations in order to have roughly the same execution time.
All queries use a fixed query size of 100.

\begin{table}
  \centering
  \sffamily
  \begin{tabular}{l l}
    \toprule
    \textbf{Multiplication factor} &
    \textbf{Dataset size} & \\

    \cline{1-2}

    500 &
    & \\

    \cline{1-2}

    1000 &
    & \\

    \cline{1-2}

    2500 &
    36 250 events &\\

    \cline{1-2}

    5000 &
    72 500 events & \\

    \bottomrule
  \end{tabular}

  \caption{Multiplication factor and dataset size}
  \label{tbl:dataset-size}
\end{table}


\subsection{Query size}
\label{subsec:query-size}

The amount of events retrieved from the database is another aspect that could possibly influence the query performance heavily.
In this section different values for the query size are compared, in order to find any trends

MongoDB repeats the following 10 000 times, while Neo4j only uses an iteration count of 10.
All queries operate on a dataset with multiplication factor of 5000, yielding 72 500 events in the database.

Queries 1, 2 and 3 in \cref{subsubsec:query-1} were used as queries in this test.
Query 4 is very similar to query 3 in structure of the queried data itself, so it was omitted.
Query 5 is a query that tests insertion of data, and it was omitted as well due to the fact that query size is not relevant for insertion queries.

\begin{figure}
  \centering
  \includegraphics[width=.8\textwidth]{img/mongodb-query-size.png}
  \caption{MongoDB query size}
  \label{fig:mongodb-query-size}
\end{figure}

The horizontal axis in figure \ref{fig:mongodb-query-size} represents query size in the logarithmic scale.
The vertical axis represents the execution time of 10 000 iterations of the query in seconds.

All measured execution times fall roughly within 7 and 9 seconds, which means that the query execution time is constant for a varying query size in MongoDB

\begin{figure}
  \centering
  \includegraphics[width=.8\textwidth]{img/neo4j-query-size.png}
  \caption{Neo4j query size}
  \label{fig:neo4j-query-size}
\end{figure}

Neo4j however, renders a completely different result.
Figure \ref{fig:neo4j-query-size} plots out the execution time for the different queries running on the Neo4j graph database management system.
The first difference with MongoDB is already very apparent in iteration size: MongoDB can handle roughly 1000 times as many requests in the same execution time.
This is due to the fact that for every entity retrieved from the database, MongoDB only has to execute one request and retrieve one document, while Neo4j's inherent linked structure means that at least three entities have to be retrieved (event, subject and item).

Query 2 and query 3 remain constant, similar to their MongoDB equivalents.
Query 1 however, rises exponentially with query size.
This is not an expected result since a database index exists for the \texttt{created\_at} field on which the query orders.
However, for small, realistic values of query size this should not pose a problem.

\subsection{Iteration count}
\label{subsec:iteration-size}

The multiplication factor in these tests was set to 5000, similar to the previous tests.
Query size for all queries is 100.

Analogous to the previous test, query 4 was not included in the benchmark.

\begin{figure}
  \centering
  \includegraphics[width=.8\textwidth]{img/mongodb-iteration-count.png}
  \caption{MongoDB iteration count}
  \label{fig:mongodb-iteration-count}
\end{figure}

\begin{figure}
  \centering
  \includegraphics[width=.8\textwidth]{img/neo4j-iteration-count.png}
  \caption{Neo4j iteration count}
  \label{fig:neo4j-iteration-count}
\end{figure}

\section{Conclusion}
\label{sec:empirical-study-conclusion}

%   Language bindings: mongo yes, couch no, neo4j yes; impedance mismatch in mongo, especially related to polymorphism

\input{chapters/09-results}
\input{chapters/10-opportunities}
\input{chapters/11-conclusion}

%%=============================================================================
%% Bijlagen
%%=============================================================================

\appendix

%%---------- Onderzoeksvoorstel -----------------------------------------------

\chapter{Research proposal}
\label{ch:research-proposal}

The subject of this thesis is based upon a research proposal that was graded by the promotor in advance. This proposal has been included as an attachment.

% Verwijzing naar het bestand met de inhoud van het onderzoeksvoorstel
%---------- Inleiding ---------------------------------------------------------

\section{Introduction} % The \section*{} command stops section numbering
\label{sec:introduction}

% problem and context
% motivation, relevance for the research
% goal and research questions

The \textcite{OpenWebslides} project provides a user-friendly platform to collaborate on webslides - slides made with modern web technologies such as HTML, CSS and JavaScript. One of the core features this application provides is \emph{co-creation}. The co-creation aspect manifests itself in several forms within the application; annotations on slides and a change suggesting system resembling GitHub's pull request feature are the main mechanisms. Because of the inherent social nature of co-creation, a basic notifications feed was also implemented. This feed is tailored to the user, and reflects the most recent changes, additions and comments relevant to the slide decks the user is interested in.

However, at the moment the functionality implemented in the system contains only the bare necessities. The module will be expanded in the future, and doing so requires a structural and conceptual rethinking of how the notifications are generated, stored and queried. The size of the dataset is also expected to grow linearly with user activity, therefore scalability is a requirement as well.

This paper has two research questions.

\begin{enumerate}
	\item What frameworks and software packages currently exist in the industry to store structured non-relational graph or document data?
	\item How is the social graph provided by the Open Webslides' notification feed structured and how is this data consumed?
\end{enumerate}

Answering these questions is paramount for the final section of the paper, which describes a data storage mechanism that performs well given the functional requirements of the project's data flow.

%---------- Use case ----------------------------------------------------------

\section{Use case}
\label{sec:use-case}

This thesis is a study of NoSQL data storage techniques applied to the \textcite{OpenWebslides} project. The online, interactive platform that the project provides includes a list of notifications in reverse chronological order, tailored to the user. This feature is called the \textit{social feed}. It enumerates the most recent social activity on the platform. For example, if a user updates a slide deck, the user's friends would be able to find a change notification in their respective social feeds.

From the perspective of the application code that generates the social feed, the user, slide deck and notification should be considered separate entities. The notification itself has relations to the other entities: the author (the \textit{subject}), the slide deck (the \textit{object}), along with the \textit{predicate} property that provides information on what operation was performed (for example creating or updating a slide deck).

This data model is also characterized by the \textit{write-once read-many} nature of the information; once the notification has been generated, it does not need to be modified again. It will also only be queried in a very specific way: the application server will always attempt to retrieve the most recent notifications starting from the user entity.
This principle is an important aspect to take into consideration in the choice of data storage mechanism.

%---------- Stand van zaken ---------------------------------------------------

\section{State-of-the-art}
\label{sec:state-of-the-art}

In current literature, studies such as \textcite{Moniruzzaman2013}, \textcite{NayakPoriyaPoojary2003} and \textcite{HammesMederoMitchell2014} have already analyzed the disparity between traditional relational database systems and NoSQL stores.
However, the conceptual and technical difference between these data storage models will not be scrutinized any further, since this paper presents a data storage solution applied to the social notification feed of the \textcite{OpenWebslides} project.

There are already many existing free and commercial products for the storage of NoSQL data, such as Redis \autocite{Redis}, CouchDB \autocite{CouchDB}, MongoDB \autocite{MongoDB} and Neo4j \autocite{Neo4j}. Finding the right database model for this use case (section \ref{sec:use-case}) is one of the hurdles this paper intends to handle. \textcite{Zhao2015} describes the development of a messaging system for astrophysical transient event notifications. Part of this paper is a qualitative comparison between document-based NoSQL storage solutions fit for this particular use case. We expect this paper to provide a solid base of reasoning in order to find a scalable and efficient solution for resolving similar computational challenges.

The goal of this paper is to provide a performant, scalable and maintainable data storage schema for the \textcite{OpenWebslides} platform, regarding the linked social graph that powers the \textit{Social Feed} functionality present in the platform.

% \autocite{KEY} => (Author, year)
% \textcite{KEY} => Author (year)

%---------- Methodologie ------------------------------------------------------
\section{Methodology}
\label{sec:methodology}

First, a range of industry-standard NoSQL database management systems such as MongoDB \autocite{MongoDB}, HBase \autocite{HBase} and Neo4j \autocite{Neo4j} will be qualitatively analyzed. Three of the five types of NoSQL database types \autocite{NayakPoriyaPoojary2003} will be included in the study: column-oriented, document based and graph databases. Criteria for comparison include how the database management system concretely stores its data on disk, the query format and specific programming language bindings. Another important aspect is the distributed nature of many NoSQL databases. Using Brewer's conjecture \autocite{Brewer2002} -- often called the CAP theorem -- the existing types of data storage systems will be examined and summarized. There is also a practical factor present in the research; this includes the license of the project, its active maintainability and future prospects.
Common types of NoSQL databases include key-value store, column-oriented, document store and graph databases \autocite{NayakPoriyaPoojary2003}. This paper will provide a short introduction to these types, before proceeding with the type that fits our use case the most.

Second, the data model specific to the Open Webslides project will be examined. We will start from the data model that is already implemented in the current iteration of the platform. At the time of writing, the existing base implementation of the social notification feed only contains two types of notifications. This paper will try to extrapolate this concept into a more generalized, abstract system in which developers can easily plug additional notification types.
The physical properties of the data model will also be taken into account: the data will be written to the data storage only once, but read many times. It is also highly interlinked information, as a notification will always relate to one or more users as a subject, and a target object as well -- most likely a slide deck or collection of slide decks. These links need to be maintained, and efficiently reconstructed when queried.

Finally, a sample dataset will be constructed using the aforementioned detailed analysis. Empirical testing will be conducted against multiple database management systems, and the results will be summarized and interpreted. Various information flows will be tested; however, the most important process remains efficiently querying the stored data.

Using the comparative study of storage engines, data model analysis and the empirical results an implementation plan will be constructed. This plan will serve as a recommendation for future development.

%---------- Verwachte resultaten ----------------------------------------------
\section{Expected results and conclusions}
\label{sec:expected_results_and_conclusions}

The NoSQL ecosystem, unlike relational databases, is headed towards specialization, so different solutions are headed in different directions \autocite{Maroo2013}. In this paper, we expect to find one type of NoSQL database that is a better fit for the Open Webslides use case, in clear contrast with the other types of storage engines. Due to the inherently highly interlinked nature of the stored data, we suspect a graph-based database management system to provide most advantages, and generally the most performant experience.

This expectation is amplified by the availability and good community support of Ruby bindings to the most popular graph database management systems.

Since the platform being discussed only caters to a small to medium user base, we do not expect the need to scale horizontally beyond one instance. However, the vertical scalability is still a topic for discussion, and we expect to determine the computational order of magnitude in order to efficiently query the given dataset during this study.

Finally, the implementation plan should describe a concrete roadmap, stretching over a development period with a baseline expectation of one to three months. Roll-out of this mechanism should also be included in this plan.

We also expect that this thesis will provide a good reference to a further stable, scalable and extendable implementation of the \textit{social feed} feature in the \textcite{OpenWebslides} project as outlined in section \ref{sec:use-case}.


%%---------- Andere bijlagen --------------------------------------------------
% TODO: Voeg hier eventuele andere bijlagen toe
%\input{...}

%%---------- Referentielijst --------------------------------------------------

\printbibliography[heading=bibintoc]
%\addcontentsline{toc}{chapter}{\textcolor{maincolor}{\IfLanguageName{dutch}{Bibliografie}{Bibliography}}}

\end{document}
