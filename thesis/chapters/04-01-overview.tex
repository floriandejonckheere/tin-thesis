\section{Overview}
\label{sec:overview}

\subsection{Relational data stores}
\ref{sec:relational-data-stores}

Relational database management systems (RDBMS) have been the de facto standard for over two decades to efficiently store and query information in a wide variety of native and web applications. These data stores are based on the relational model devised by Edgar Codd. The data in the system is presented as relations: a collection of data consisting of rows and columns. The user of the database can query and manipulate the data using relational operators.

Codd presented thirteen rules for a database in order for it to be considered a \textit{relational database}. However, many of the modern database systems do not adhere strictly to all of these rules. More commonly, a relational database is defined as a database that exposes its information using a collection of rows and columns.

Relational database management systems provide certain guarantees during database transactions. Reuter and H\"arder introduced the concept of ACID: an acronym referring to a set of transactional properties.

\begin{enumerate}
  \item \textbf{Atomicity} \\ Each transaction is ``all or nothing'', either the transaction completes successfully and the data is mutated in an atomic way, or the transaction fails entirely and none of the data in the database is modified.
  \item \textbf{Consistency} \\ Each transaction, when successful, only commits legal results. This entails that the database is always in a consistent state.
  \item \textbf{Isolation} \\ Actions within a single transaction are not visible to other, concurrently running transactions.
  \item \textbf{Durability} \\ Once a transaction has completed successfully and been committed to the database, the system must guarantee that these results survive any subsequent malfunction.
\end{enumerate}

\subsection{NoSQL data stores}
\label{sec:nosql-data-stores}

In the early 2000s, a totally different concept of data storage was popularized. NoSQL data stores provide a system of storage that is ``non SQL'' or ``non relational''. More recently, the NoSQL denomination has also been explained as ``not only SQL'', pointing on the fact that some data stores provide a different interface besides SQL.

The use of non-relational data stores was motivated by the needs of Web 2.0 companies such as Facebook and Google. NoSQL provides a way to store massive amounts of data in a flexible way. The architecture of such systems is usually more simple, and focuses more on horizontal as opposed to vertical scalability. Data is not stored in rows and columns as is the case in the relational model, but rather in a different data structure. \textcite{Nayak2013} divides the data models used by NoSQL data stores into five categories.

\subsubsection{Key-value}
\label{sec:key-value}

Key-value data stores are simple in design, yet powerful and efficient when used correctly. A key-value data store allows the user to store schemaless data using a plain, unique key, creating a \textit{key} and \textit{value} pair. Similar to hash tables, the values are stored and queried using the provided key which is used as an index. Modern key-value data stores prefer high scalability over consistency, entailing that more advanced ad-hoc querying and analytical operations on the data such as joins are not available.

\subsubsection{Column-oriented}
\label{sec:column-oriented}

Column-oriented data stores are designed to store data by column rather than by row. This kind of NoSQL data store is more similar to relational database systems than the other categories. In column-oriented data stores, each key is associated with a set of attributes, stored in columns. Concretely this means that the data is indexed by column value, rather than by row.

\subsubsection{Document}
\label{sec:document}

Document databases store their data in documents, indexed by a unique key. This is technically a subclass of key-value stores, however the difference lies in the interpretation of the data. Since documents are schemaless, each document may have a comparable structure, or a completely different one. In contrast with key-value data stores, the value (document) is not opaque to the database management system, but is interpreted and used for query optimization.

\subsubsection{Graph}
\label{sec:graph}

As the name suggests, graph data stores keep their data in the form of a graph. The graph is made up of nodes and edges, the former being the entities containing the data and the latter being the relations between these entities. Graph data stores use a technique called \textit{index-free adjacency}, where every node contains a logical pointer to the adjacent node. This makes graph traversal a very fast operation. Graph databases are ACID compliant.

\subsubsection{Object-oriented}
\label{sec:object-oriented}

Object-oriented data stores represent the data as an object, closely resembling the concept of an object in object-oriented programming languages. This puts object-oriented data stores much closer to the programming environment than other database systems. It provides all features inherent to object-oriented languages, such as data encapsulation, inheritance and polymorphism. The concepts of class, class attributes and an object can be mapped onto the relational concepts of a table, columns and a tuple. This concept of data storage makes software development more flexible.

% BASE
% CAP, ACID vs BASE
% NewSQL

% Introduction to NoSQL
%   Data models
%   ACID vs BASE
%   Document, Column, Key-Value, Graph, Object Oriented, NewSQL
%   Multi-model: OrientDB
%   Querying: REST/SQL, MapReduce
%   Scaling
%   Security
