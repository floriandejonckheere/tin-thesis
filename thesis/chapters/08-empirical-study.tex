\chapter{Empirical study}
\label{ch:empirical-study}

In this chapter we will implement the previously proposed physical data models for the MongoDB, CouchDB and Neo4j data stores.
In this context, a Ruby on Rails application was developed that uses three libraries that provide language bindings to the respective data stores.
% TODO: references
A custom benchmarking framework was implemented using the RSpec library to provide structure and maintainability and the \texttt{Benchmark} class built into the Ruby implementation to measure query execution.

\section{Testing setup}
\label{sec:testing-setup}

In order to keep the results of the tests consistent, the following rules are applied:

\begin{itemize}
  \item Query caching is disabled
  \item Connection pooling is disabled
  \item Clustering is disabled
  \item No additional performance tweaks were applied on the database management systems
  \item All disk persistence is mapped to memory on a RAM disk, transparent to the database management systems
\end{itemize}

Native protocols were chosen for every data store: Mongo Wire Protocol for MongoDB, HTTP for CouchDB and Bolt for Neo4j.

All benchmarks were performed on a single machine with the following specifications:

\begin{itemize}
  \item Intel Core i7-3840QM (4 cores, 8 threads), 45W TDP
  \item Hyperthreading and \TODO{Intel Turbo Boost} enabled
  \item 32GB DDR4 RAM
  \item 180GB SATA-III SSD
\end{itemize}

The following software versions were used:

\begin{itemize}
  \item Arch Linux (rolling release)
  \item MongoDB 3.6.3
  \item CouchDB 2.1.1
  \item Neo4j Community Edition 3.0.6
  \item Ruby 2.5.0
\end{itemize}

\section{Procedure}
\label{sec:procedure}

In order to compare the performance of the three data stores, the following procedure was followed.

First, the database was filled with random testing data.
The amount of testing data linearly depends on a multiplication factor, which \TODO{varied} throughout the benchmark tests.

Next, the full test suite was ran sequentially for the three data stores, and the timing results were written to separate files.
It is important to note that every reference query was executed many times, depending on different factors.
Since the execution time of a single iteration is negligable, every query was executed a number of times to negate the effect of external factors, such as operating system scheduling and I/O wait times.
We have chosen the iteration counts of \TODO{1 000, 10 000 and 100 000}.
Having multiple iteration runs gives us the opportunity to analyze the vertical scalability of the query as well.
As indicated in the informal descriptions of the reference queries, the query itself is also dependent on a variable $N$ which signifies the event count that is to be retrieved from the database.
We believe a sane default for this value in a concrete implementation would be 100, meaning that 100 events will be retrieved on every query run.
However, in the test suite this variable takes the value of \TODO{100, 1 000 and 10 000} to allow for more reliable timings.

This gives us 9 timing results for every query for every data store in total.
These results are analyzed, graphed and discussed in the sections below.

\section{Previous work}
\label{sec:previous-work}

% Netflix benchmark, http://www.bgbenchmark.org/BG/, Yahoo! Cloud Serving Benchmark (YCSB)

\section{Conclusion}
\label{sec:comparative-study-conclusion}
