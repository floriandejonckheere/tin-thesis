%%=============================================================================
%% Inleiding
%%=============================================================================

\chapter*{Introduction}
\label{ch:introduction}

De inleiding moet de lezer net genoeg informatie verschaffen om het onderwerp te begrijpen en in te zien waarom de onderzoeksvraag de moeite waard is om te onderzoeken. In de inleiding ga je literatuurverwijzingen beperken, zodat de tekst vlot leesbaar blijft. Je kan de inleiding verder onderverdelen in secties als dit de tekst verduidelijkt. Zaken die aan bod kunnen komen in de inleiding~\autocite{Pollefliet2011}:

\begin{itemize}
  \item context, achtergrond
  \item afbakenen van het onderwerp
  \item verantwoording van het onderwerp, methodologie
  \item probleemstelling
  \item onderzoeksdoelstelling
  \item onderzoeksvraag
  \item \ldots
\end{itemize}

\section{Context}
\label{sec:context}

In this age of computers and smartphones, older and deprecated methods of teach are quickly being replaced by the digital equivalent. Course content has shifted from being printed in full-text on paper, to static slides on an overhead projector and to the digital screen present in every modern classroom. However, there is a lot more potential to gain from the modernization of course content. Education is still too often a one-way street, where students are obligated to process the course content without being able to provide much challenge or activity. This also does not allow for any dialogue to take place between students and teachers concerning feedback and improvement of the material itself.

Technological constraints in current iterations of educational software do not allow this co-creation discourse easily. Material being locked to specific versions of proprietary software is just one of the many problems teachers might encounter when trying to apply this concept in real life.

By utilizing interactive tools and applications, the teacher can engage the students more directly.

By building on modern, open standards, the Open Webslides project \textcite{OpenWebslides2017} aims to provide a platform that solves these problems. It creates a user-friendly environment where teachers can create courses based on open source technologies and standards, and it allows them to apply the co-creation narrative easily. This also enables users to share their material not just with their immediate environment but with a much broader educational audience.

\section{Problem statement}
\label{sec:problem-statement}

Uit je probleemstelling moet duidelijk zijn dat je onderzoek een meerwaarde heeft voor een concrete doelgroep. De doelgroep moet goed gedefinieerd en afgelijnd zijn. Doelgroepen als ``bedrijven,'' ``KMO's,'' systeembeheerders, enz.~zijn nog te vaag. Als je een lijstje kan maken van de personen/organisaties die een meerwaarde zullen vinden in deze bachelorproef (dit is eigenlijk je steekproefkader), dan is dat een indicatie dat de doelgroep goed gedefinieerd is. Dit kan een enkel bedrijf zijn of zelfs één persoon (je co-promotor/opdrachtgever).

\section{Research question}
\label{sec:research-question}

Wees zo concreet mogelijk bij het formuleren van je onderzoeksvraag. Een onderzoeksvraag is trouwens iets waar nog niemand op dit moment een antwoord heeft (voor zover je kan nagaan). Het opzoeken van bestaande informatie (bv. ``welke tools bestaan er voor deze toepassing?'') is dus geen onderzoeksvraag. Je kan de onderzoeksvraag verder specifiëren in deelvragen. Bv.~als je onderzoek gaat over performantiemetingen, dan

\section{Research goal}
\label{sec:research-goal}

Wat is het beoogde resultaat van je bachelorproef? Wat zijn de criteria voor succes? Beschrijf die zo concreet mogelijk.

\section{Thesis objective}
\label{sec:thesis-objective}

% Het is gebruikelijk aan het einde van de inleiding een overzicht te
% geven van de opbouw van de rest van de tekst. Deze sectie bevat al een aanzet
% die je kan aanvullen/aanpassen in functie van je eigen tekst.

De rest van deze bachelorproef is als volgt opgebouwd:

In Hoofdstuk~\ref{ch:state-of-the-art} wordt een overzicht gegeven van de stand van zaken binnen het onderzoeksdomein, op basis van een literatuurstudie.

In Hoofdstuk~\ref{ch:methodology} wordt de methodologie toegelicht en worden de gebruikte onderzoekstechnieken besproken om een antwoord te kunnen formuleren op de onderzoeksvragen.

% TODO: Vul hier aan voor je eigen hoofstukken, één of twee zinnen per hoofdstuk

In Hoofdstuk~\ref{ch:conclusion}, tenslotte, wordt de conclusie gegeven en een antwoord geformuleerd op de onderzoeksvragen. Daarbij wordt ook een aanzet gegeven voor toekomstig onderzoek binnen dit domein.

