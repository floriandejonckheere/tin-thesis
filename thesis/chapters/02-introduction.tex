%%=============================================================================
%% Introduction
%%=============================================================================

\chapter*{Introduction}
\label{ch:introduction}

%~ De inleiding moet de lezer net genoeg informatie verschaffen om het onderwerp te begrijpen en in te zien waarom de onderzoeksvraag de moeite waard is om te onderzoeken. In de inleiding ga je literatuurverwijzingen beperken, zodat de tekst vlot leesbaar blijft. Je kan de inleiding verder onderverdelen in secties als dit de tekst verduidelijkt. Zaken die aan bod kunnen komen in de inleiding~\autocite{Pollefliet2011}:

%~ \begin{itemize}
  %~ \item context, achtergrond
  %~ \item afbakenen van het onderwerp
  %~ \item verantwoording van het onderwerp, methodologie
  %~ \item probleemstelling
  %~ \item onderzoeksdoelstelling
  %~ \item onderzoeksvraag
  %~ \item \ldots
%~ \end{itemize}

\section{Context}
\label{sec:context}

In this age of computers and smartphones, older and deprecated methods of teaching are quickly being replaced by the digital equivalent. Course content has shifted from being printed in full-text on paper, to static slides on an overhead projector, to the digital screen present in every modern classroom. However, there is a lot more potential to gain from the modernization of course content. Education is still too often a one-way street, where students are obligated to process the course content without being provided much challenge or activity. This does not allow for any dialogue to take place between students and teachers concerning feedback and improvement of the course material itself.

Technological constraints in current iterations of educational software do not allow this co-creation discourse easily. Material being locked to specific versions of proprietary software is just one of the many problems teachers might encounter when trying to apply this concept in real life.

By utilizing interactive tools and applications, the teacher can engage the students more directly.

By building on modern, open standards, the Open Webslides project \autocite{OpenWebslides2017} aims to provide a platform that solves these problems. It creates a user-friendly environment where teachers can create courses based on open source technologies and standards, and it allows them to apply the co-creation narrative easily. This also enables users to share their material not just with their immediate environment but with a much broader educational audience.

\section{Problem statement}
\label{sec:problem-statement}

The Open Webslides platform incorporates several ways to stimulate spontaneous co-creation between teachers and students. One of the most prominent elements is the \textit{Recent Activity} feed. This reverse chronologically ordered list enumerates the most recent user interactions with the platform and with other users. Feed items range from simple actions such as a user having created or modified course content, to more complex social interactions like a discussion being held using annotations or a student's changes being incorporated in a teacher's courses.

The size of the activity feed data set is directly correlated to the size and activity of the userbase. It has the potential to grow explosively, especially in timespans critical to teachers and students such as examination periods. In order for the infrastructure to be able to handle the deluge of the queried information, designing a system that allows for efficient querying and easy scalability is of paramount importance. Further decoupling of this subsystem from the business-critical processes is also important to ensure that downtime of this subsystem has no impact on round the clock availability of course content to the users.

This research thesis will provide a comparative and analytical study of existing NoSQL database solutions for the Open Webslides project to implement an efficient, scalable data storage system in the context of the Recent Activity feed already incorporated into the platform.

\section{Research questions}
\label{sec:research-questions}

% Research question 1: What frameworks and software packages currently exist in the industry to store structured non-relational graph or document data?
% Research question 2: How is the social graph provided by the Open Webslides’ recent activity feed structured and how is this data consumed?
% Research question 3: What NoSQL database is most fit for storing this social graph?

\section{Research goal and objectives}
\label{sec:research-goal-and-objectives}

% Research goal 1: Providing overview of non-relational data stores
% Research goal 2: Providing suggestion of data store for Open Webslides project

\clearpage{}

In the chapter \autoref{ch:state-of-the-art} an overview will be presented of current research, applied solutions and other related work in the research domain based on a literature study.

The chapter \ref{ch:methodology} will clarify the analytical research approaches used in this thesis to attempt to formulate an answer to the research questions.

Finally, the \ref{ch:conclusion} chapter will present a conclusion and an applied answer to the research questions.
