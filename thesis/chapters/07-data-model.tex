\chapter{Data model}
\label{ch:data-model}

In this section we describe the conceptual domain of this research, and provide a logical data model that fits the domain.
For every NoSQL data model selected, we also propose a physical data model.
This data model will then be implemented in a Ruby on Rails application using the available language bindings.
Finally, a set of reference queries will be proposed, queries that may typically be used in the context of the Recent Activity feed.

\section{Domain description}
\label{sec:domain-description}

On the Open Webslides platform, there is technically no distinction between a teacher's and a student's account.
Both of them are represented by the \texttt{User} entity in the database.
This entity contains information pertaining to the user, such as email address, first name and last name.
The data models described in the next sections will closely follow the existing data model of the platform.
However, certain irrelevant attributes are omitted from the model in order to improve readability and simplify abstraction.

A user can create or modify course content in an interactive online editor.
The actual course content, formally called a topic, is stored inside a git repository on the filesystem.
However, the platform also maintains a record of topic metadata in the relational database.
This metadata includes title and description, but also permissions and contributors on the course content.
Per illustration, the \texttt{title} and \texttt{description} attributes are included in the data models.

From a technological perspective, the user has three distinct paths of action for (co-)creation on course content.
Since the permission model in the platform is not relevant to this research, we will not go into detail on it.
First, the user can directly modify the course content if the user has permission to do so.
The second option is creating annotations on the topic.
This allows the user to attach private or public notes to specific content on the topic.
Annotations are stored in the relational database, mapped to the \texttt{Annotation} entity.
The entity contains a logical pointer to the annotated content, which is omitted from the data models.
The final possibility to integrate user content into a topic is by adding comments.
In contrast to annotations, comments have a typical structure.
They can take the form of questions, notes, suggestions, and can also be nested - which allows simple interaction and conversation between multiple users, and effectively enables dialogue between students and teachers.

Consider the following domain description structured as Recent Activity events.

``
\textbf{John} created \textbf{Topic A}.
''

``
\textbf{Jane} commented on \textbf{Topic B}:
\textit{This is not a good example. Try and find a better one.}
''

``
\textbf{John} commented on \textbf{Jane}'s comment on \textbf{Topic B}:
\textit{I agree.}
''

``
\textbf{Jane} annotated \textbf{Topic A}.
''

``
\textbf{Bob} reacted to \textbf{John}'s comment on \textbf{Topic B}.
''

Using this description of the domain, we can start to derive logical and physical data models for the selected NoSQL data stores.

\section{Physical data model}
\label{sec:physical-data-model}

\subsection{Document-oriented data model}
\label{subsec:document-data-model}

\subsection{Column-oriented data model}
\label{subsec:column-data-model}

\subsection{Graph-oriented data model}
\label{subsec:graph-data-model}
