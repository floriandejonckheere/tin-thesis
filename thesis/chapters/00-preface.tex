%%=============================================================================
%% Preface
%%=============================================================================

\chapter*{Preface}
\label{ch:preface}

%% TODO:
%% Het voorwoord is het enige deel van de bachelorproef waar je vanuit je
%% eigen standpunt (``ik-vorm'') mag schrijven. Je kan hier bv. motiveren
%% waarom jij het onderwerp wil bespreken.
%% Vergeet ook niet te bedanken wie je geholpen/gesteund/... heeft

The idea for this research originally comes from the Open Webslides project and its many little side activities in development.
One of the things that has always fascinated me was how the platform would handle a massive influx of users, and specifically how it would relate to the non-critical data storage in the Recent Activity feed.
I wanted to find out how a NoSQL data store would be integrated into the flow of information, and what kind of data store would be the most efficient, scalable solution for this problem.
My interest was also piqued by the usage of a graph database (Neo4j) in a personal project, and how the data of the Open Webslides project would fit into the graph theoretical model.
Digging into this subject while still maintaining my vision on the Ruby on Rails implementation in the platform allowed me to let the question bloom into this research thesis.

This thesis was in part achieved by the support of my promotor, who has given me many tips and tricks, and provided a framework for conducting a proper research.
My co-promotor also had an important influence on certain decisions taken in the research and development phase, being a person who is immersed in the relational and non-relational academic database world.
Finally, my friends and family also deserve recognition for helping me accomplish this paper, which is the culmination of three years higher education in a fast-moving and innovative field.
