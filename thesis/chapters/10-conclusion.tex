%%=============================================================================
%% Conclusie
%%=============================================================================

\chapter{Conclusion}
\label{ch:conclusion}

%% TODO: Trek een duidelijke conclusie, in de vorm van een antwoord op de
%% onderzoeksvra(a)g(en). Wat was jouw bijdrage aan het onderzoeksdomein en
%% hoe biedt dit meerwaarde aan het vakgebied/doelgroep? Reflecteer kritisch
%% over het resultaat. Had je deze uitkomst verwacht? Zijn er zaken die nog
%% niet duidelijk zijn? Heeft het onderzoek geleid tot nieuwe vragen die
%% uitnodigen tot verder onderzoek?

The intended aim of NoSQL data stores is to store and process massive amounts of data in a distributed fashion.
However, there exist multiple data models which store data in a different way, which can be more or less performant to the applied use case.
NoSQL data stores are not ``one size fits all''.

In this paper all proposed research questions were satisfactorily answered.
First, a literature review offered insight in the NoSQL products currently available on the open source and commercial market.
The different data models were discussed by advantages and disadvantages.

Second, a conceptual and technical definition of the Open Webslides data model was designed and analyzed.
This proved to be a valuable resource to proceed with the design and development of the physical data models and corresponding implementation in the MongoDB and Neo4j graph data stores.
The second research question was answered by these presented data models, along with the five reference queries that were drawn up and represent typical usage in the studied use case.

Finally, as most efficient data store MongoDB was selected as NoSQL candidate.
Subsequently, the recommendation for the Open Webslides development team is to use the MongoDB as a polyglot persistent NoSQL database to store the derived data for the Recent Activity feed.

In contrast to the expected results, using the Neo4j data store turned out to be magnitudes slower in performing the reference queries on the developed data model.
Instead, MongoDB demonstrated that the document data model is far superior over the graph data model in this specific use case.
The fact that the data from which the Recent Activity feed is derived is highly interlinked and easily structured as a graph, counteracts the expectations of a highly linked data structure to be efficient.

It was found that MongoDB has excellent support for Ruby and Ruby on Rails using the Mongoid ORM framework.
For all these reasons MongoDB is the clear winner as performant NoSQL data store for the Open Webslides' Recent Activity feed.

This research presents a case study of NoSQL data store selection for a specific use case.
The paper may offer additional opportunities for research as described in \cref{ch:opportunities}, and can be used as a guideline to pick the right data store for other, similar use cases.

Finally, the physical implementation of the data schema proposed in this thesis can be implemented in the Open Webslides platform to integrate a highly scalable, performant polyglot infrastructure to accommodate the Recent Activity feed.
